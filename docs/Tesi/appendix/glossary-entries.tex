% Definizione degli acronimi

% Definizione del glossario (senza duplicati)
\newglossaryentry{cybersecurity} {
    name={Cybersecurity},
    text={cybersecurity},
    sort={cybersecurity},
    description={L'insieme delle azioni volte a difendere computer, server, dispositivi mobili, sistemi elettronici, reti e dati dalle minacce informatiche}
}

\newglossaryentry{overfitting} {
    name={overfitting},
    text={overfitting},
    sort={overfitting},
    description={definizone}
}

\newglossaryentry{adware} {
    name={Adware},
    text={adware},
    sort={adware},
    description={Software progettato per visualizzare pubblicità indesiderata sul dispositivo dell'utente, spesso utilizzato per generare entrate pubblicitarie.}
}

\newglossaryentry{backdoor} {
    name={Backdoor},
    text={backdoor},
    sort={backdoor},
    description={Un metodo segreto per bypassare i normali meccanismi di autenticazione, consentendo l'accesso non autorizzato a un sistema.}
}

\newglossaryentry{downloader} {
    name={Downloader},
    text={downloader},
    sort={downloader},
    description={Un tipo di malware progettato per scaricare e installare altri malware su un sistema compromesso.}
}

\newglossaryentry{ransomware} {
    name={Ransomware},
    text={ransomware},
    sort={ransomware},
    description={Un tipo di malware che cripta i dati della vittima, richiedendo un riscatto per ripristinarli.}
}

\newglossaryentry{spyware} {
    name={Spyware},
    text={spyware},
    sort={spyware},
    description={Software malevolo utilizzato per raccogliere dati sensibili dell'utente senza il suo consenso.}
}

\newglossaryentry{trojan} {
    name={Trojan},
    text={trojan},
    sort={trojan},
    description={Un malware che si presenta come un software legittimo, inducendo gli utenti a installarlo.}
}

\newglossaryentry{virus} {
    name={Virus},
    text={virus},
    sort={virus},
    description={Un tipo di malware che si replica infettando file o programmi, diffondendosi ad altri sistemi.}
}

\newglossaryentry{gan} {
    name={Generative Adversarial Networks (GAN)},
    text={GAN},
    sort={gan},
    description={Una classe di algoritmi di machine learning in cui due reti neurali, un generatore e un discriminatore, competono per migliorare le loro prestazioni.}
}

\newglossaryentry{cnn} {
    name={Convolutional Neural Network (CNN)},
    text={CNN},
    sort={cnn},
    description={Una rete neurale progettata per l'elaborazione di dati strutturati come immagini, utilizzando strati convoluzionali per estrarre caratteristiche.}
}

\newglossaryentry{tfidf} {
    name={TF-IDF (Term Frequency - Inverse Document Frequency)},
    text={TF-IDF},
    sort={tfidf},
    description={Una tecnica per quantificare l'importanza di termini in un dataset testuale, bilanciando la frequenza locale e globale.}
}

\newglossaryentry{pca} {
    name={PCA (Principal Component Analysis)},
    text={PCA},
    sort={pca},
    description={Una tecnica di riduzione della dimensionalità che identifica le componenti principali di un dataset, mantenendo la massima varianza.}
}

\newglossaryentry{minmaxscaling} {
    name={Min-Max Scaling},
    text={min-max scaling},
    sort={minmaxscaling},
    description={Una tecnica di normalizzazione dei dati che ridimensiona i valori in un intervallo specifico, spesso tra 0 e 1.}
}

\newglossaryentry{shap} {
    name={SHAP (SHapley Additive exPlanations)},
    text={SHAP},
    sort={shap},
    description={Un metodo per interpretare i modelli di machine learning, assegnando valori di contributo a ciascuna caratteristica per una predizione.}
}

\newglossaryentry{dropout} {
    name={Dropout},
    text={dropout},
    sort={dropout},
    description={Una tecnica di regolarizzazione utilizzata nelle reti neurali per prevenire l'overfitting, disattivando casualmente neuroni durante l'addestramento.}
}

\newglossaryentry{adam} {
    name={Adam Optimizer},
    text={Adam},
    sort={adam},
    description={Un algoritmo di ottimizzazione che combina i benefici di AdaGrad e RMSProp, adattando i tassi di apprendimento per ciascun parametro.}
}

\newglossaryentry{filelessmalware} {
    name={Fileless Malware},
    text={fileless malware},
    sort={filelessmalware},
    description={Un tipo di malware che non utilizza file tradizionali per eseguire il proprio codice, operando direttamente nella memoria del sistema.}
}

\newglossaryentry{scareware} {
    name={Scareware},
    text={scareware},
    sort={scareware},
    description={Software progettato per spaventare l'utente con falsi avvisi di sicurezza, inducendolo a compiere azioni dannose, come acquistare software fraudolento.}
}

\newglossaryentry{rat} {
    name={RAT (Remote Access Trojan)},
    text={RAT},
    sort={rat},
    description={Un tipo di Trojan che consente a un utente remoto non autorizzato di controllare completamente un dispositivo infetto.}
}

\newglossaryentry{malvertising} {
    name={Malvertising},
    text={malvertising},
    sort={malvertising},
    description={L'uso di pubblicità online legittime per diffondere malware tramite link o contenuti incorporati.}
}

\newglossaryentry{cryptojacker} {
    name={Cryptojacker},
    text={cryptojacker},
    sort={cryptojacker},
    description={Malware progettato per sfruttare le risorse del dispositivo infetto per minare criptovalute senza il consenso dell'utente.}
}

\newglossaryentry{logicbomb} {
    name={Logic Bomb},
    text={logic bomb},
    sort={logicbomb},
    description={Un tipo di malware che si attiva quando vengono soddisfatte determinate condizioni predefinite.}
}

\newglossaryentry{dynamicanalysis} {
    name={Dynamic Analysis},
    text={dynamic analysis},
    sort={dynamicanalysis},
    description={Una tecnica di analisi del software che studia il comportamento di un programma durante la sua esecuzione.}
}

\newglossaryentry{staticanalysis} {
    name={Static Analysis},
    text={static analysis},
    sort={staticanalysis},
    description={Una tecnica di analisi del software che esamina il codice sorgente o binario senza eseguire il programma.}
}

\newglossaryentry{reverseengineering} {
    name={Reverse Engineering},
    text={reverse engineering},
    sort={reverseengineering},
    description={Il processo di analisi di un sistema software o hardware per comprenderne la struttura, le funzionalità e il funzionamento.}
}

\newglossaryentry{featureextraction} {
    name={Feature Extraction},
    text={feature extraction},
    sort={featureextraction},
    description={Il processo di trasformazione dei dati grezzi in un insieme di caratteristiche utili per l'addestramento di modelli di machine learning.}
}

\newglossaryentry{confusionmatrix} {
    name={Confusion Matrix},
    text={confusion matrix},
    sort={confusionmatrix},
    description={Una tabella utilizzata per valutare le prestazioni di un modello di classificazione, confrontando i valori predetti con quelli reali.}
}

\newglossaryentry{specificity} {
    name={Specificity},
    text={specificity},
    sort={specificity},
    description={Una metrica che misura la capacità di un modello di identificare correttamente i negativi veri in un problema di classificazione binaria.}
}

\newglossaryentry{blackboxmodel} {
    name={Blackbox},
    text={blackbox},
    sort={blackbox},
    description={Un modello di machine learning il cui funzionamento interno non è trasparente o comprensibile, ma è valutato solo in base agli input e agli output.}
}

\newglossaryentry{datavisualization} {
    name={Data Visualization},
    text={data visualization},
    sort={datavisualization},
    description={La rappresentazione visiva di dati per facilitarne la comprensione e l'interpretazione.}
}

\newglossaryentry{hiddenlayers} {
    name={Hidden Layers},
    text={hidden layers},
    sort={hiddenlayers},
    description={Strati interni di una rete neurale che elaborano i dati tra l'input e l'output, apprendendo rappresentazioni complesse.}
}

\newglossaryentry{neurons} {
    name={Neurons},
    text={neurons},
    sort={neurons},
    description={Unità fondamentali di elaborazione in una rete neurale, responsabili dell'elaborazione e della trasmissione delle informazioni.}
}
