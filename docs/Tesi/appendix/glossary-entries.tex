% Definizione degli acronimi

% Definizione del glossario (senza duplicati)
\newglossaryentry{cybersecurity} {
    name={Cybersecurity},
    text={cybersecurity},
    sort={cybersecurity},
    description={L'insieme delle azioni volte a difendere computer, server, dispositivi mobili, sistemi elettronici, reti e dati dalle minacce informatiche}
}

\newglossaryentry{overfitting} {
    name={Overfitting},
    text={overfitting},
    sort={overfitting},
    description={ Una condizione in cui un modello di machine learning si adatta troppo ai dati di addestramento, riducendo le prestazioni sui dati di test.}
}

\newglossaryentry{gan} {
    name={Generative Adversarial Networks (GAN)},
    text={Generative Adversarial Network},
    sort={gan},
    description={Una classe di algoritmi di machine learning in cui due reti neurali, un generatore e un discriminatore, competono per migliorare le loro prestazioni.}
}

\newglossaryentry{cnn} {
    name={Convolutional Neural Network (CNN)},
    text={CNN},
    sort={cnn},
    description={Una rete neurale progettata per l'elaborazione di dati strutturati come immagini, utilizzando strati convoluzionali per estrarre caratteristiche.}
}

\newglossaryentry{tfidf} {
    name={TF-IDF (Term Frequency - Inverse Document Frequency)},
    text={TF-IDF},
    sort={tfidf},
    description={Una tecnica per quantificare l'importanza di termini in un dataset testuale, bilanciando la frequenza locale e globale.}
}

\newglossaryentry{pca} {
    name={PCA (Principal Component Analysis)},
    text={PCA},
    sort={pca},
    description={Una tecnica di riduzione della dimensionalità che identifica le componenti principali di un dataset, mantenendo la massima varianza.}
}

\newglossaryentry{minmaxscaling} {
    name={Min-Max Scaling},
    text={min-max scaling},
    sort={minmaxscaling},
    description={Una tecnica di normalizzazione dei dati che ridimensiona i valori in un intervallo specifico, spesso tra 0 e 1.}
}

\newglossaryentry{shap} {
    name={SHAP (SHapley Additive exPlanations)},
    text={SHAP},
    sort={shap},
    description={Un metodo per interpretare i modelli di machine learning, assegnando valori di contributo a ciascuna caratteristica per una predizione.}
}

\newglossaryentry{dropout} {
    name={Dropout},
    text={dropout},
    sort={dropout},
    description={Una tecnica di regolarizzazione utilizzata nelle reti neurali per prevenire l'overfitting, disattivando casualmente neuroni durante l'addestramento.}
}

\newglossaryentry{adam} {
    name={Adam Optimizer},
    text={Adam},
    sort={adam},
    description={Un algoritmo di ottimizzazione che combina i benefici di AdaGrad e RMSProp, adattando i tassi di apprendimento per ciascun parametro.}
}

\newglossaryentry{explainability}{
    name={Explainability},
    text={explainability},
    sort={explainability},
    description={La capacità di spiegare in modo chiaro e comprensibile il funzionamento e le decisioni di un modello di machine learning.}
}

\newglossaryentry{batchsize}{
    name={Batch Size},
    text={batch size},
    sort={batchsize},
    description={Il numero di campioni utilizzati per aggiornare i pesi di un modello di machine learning durante l'addestramento.}
}

\newglossaryentry{hyperparameter}{
    name={Hyperparameter},
    text={iperparametri},
    sort={hyperparameter},
    description={Un parametro che controlla il processo di addestramento di un modello di machine learning, non appreso direttamente dai dati.}
}

\newglossaryentry{lossfunction}{
    name={Loss Function},
    text={loss function},
    sort={lossfunction},
    description={Una funzione che misura l'errore tra i valori predetti e quelli reali di un modello di machine learning, utilizzata per ottimizzare i pesi.}
}

\newglossaryentry{dynamicanalysis} {
    name={Dynamic Analysis},
    text={analisi dinamica},
    sort={dynamicanalysis},
    description={Una tecnica di analisi del software che studia il comportamento di un programma durante la sua esecuzione.}
}

\newglossaryentry{staticanalysis} {
    name={Static Analysis},
    text={analisi statica},
    sort={staticanalysis},
    description={Una tecnica di analisi del software che esamina il codice sorgente o binario senza eseguire il programma.}
}

\newglossaryentry{reverseengineering} {
    name={Reverse Engineering},
    text={reverse engineering},
    sort={reverseengineering},
    description={Il processo di analisi di un sistema software o hardware per comprenderne la struttura, le funzionalità e il funzionamento.}
}

\newglossaryentry{featureextraction} {
    name={Feature Extraction},
    text={feature extraction},
    sort={featureextraction},
    description={Il processo di trasformazione dei dati grezzi in un insieme di caratteristiche utili per l'addestramento di modelli di machine learning.}
}

\newglossaryentry{confusionmatrix} {
    name={Confusion Matrix},
    text={matrice di confusione},
    sort={confusionmatrix},
    description={Una tabella utilizzata per valutare le prestazioni di un modello di classificazione, confrontando i valori predetti con quelli reali.}
}

\newglossaryentry{blackboxmodel} {
    name={Blackbox},
    text={blackbox},
    sort={blackbox},
    description={Un modello di machine learning il cui funzionamento interno non è trasparente o comprensibile, ma è valutato solo in base agli input e agli output.}
}

\newglossaryentry{datavisualization} {
    name={Data Visualization},
    text={data visualization},
    sort={datavisualization},
    description={La rappresentazione visiva di dati per facilitarne la comprensione e l'interpretazione.}
}

\newglossaryentry{hiddenlayers} {
    name={Hidden Layers},
    text={hidden layers},
    sort={hiddenlayers},
    description={Strati interni di una rete neurale che elaborano i dati tra l'input e l'output, apprendendo rappresentazioni complesse.}
}

\newglossaryentry{neurons} {
    name={Neurons},
    text={neurons},
    sort={neurons},
    description={Unità fondamentali di elaborazione in una rete neurale, responsabili dell'elaborazione e della trasmissione delle informazioni.}
}
