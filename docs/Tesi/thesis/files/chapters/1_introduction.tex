\chapter{Introduzione}
\label{chap:introduzione}

% \begin{figure}[H]
%     \centering
%     \includegraphics[alt={Testo alternativo dell'immagine}, width=1\columnwidth]{img/quantum_entanglement.jpeg}
%     \caption{Lorem}
%     \label{fig:entanglement}
% \end{figure}

% Lorem Figure \ref{fig:entanglement}

% Esempio di utilizzo di un termine nel glossario \gls{api}.

% Esempio di citazione direttamente nel testo \cite{site:agile-manifesto}.

% Esempio di citazione nel piè di pagina \footcite{womak:lean-thinking}.


% \section{Lo SPRITZ group}
% Lo stage in questa tesi discusso è stato svolto in collaborazione con il gruppoSPRITZ dell’Università degli Studi di Padova. Questo gruppo, il cui nome è acronimodi Security and PRIvacy Through Zeal, è nato nel 2011 ed è guidato dal Prof. ContiMauro con lo scopo di accogliere ricercatori che vogliano contribuire allo sviluppo dinuove tecnologie per la sicurezza e la privacy


% \section{L'idea}
% L'obiettivo principale dello stage consiste nello sviluppo di un sistema avanzato per la rilevazione di malware attraverso l'uso delle Reti Generative Avversarie (GAN). Questo approccio si prefigge di esplorare e sperimentare le potenzialità delle GAN nell'identificare e generare varianti di malware, migliorando la capacità di rilevazione oltre i limiti delle tecniche tradizionali. Durante lo stage, sarà fondamentale la collaborazione con esperti del settore per affinare le tecniche di deep learning applicate e per valutare l'efficacia del sistema in scenari reali. Il progetto mira a coniugare teoria e pratica, utilizzando dataset reali e strumenti di ultima generazione per un'applicazione diretta delle conoscenze acquisite nel campo dell'intelligenza artificiale e della cybersecurity.


% \section{Organizzazione del testo}
% Il presente documento è strutturato per offrire una panoramica completa e dettagliata del lavoro di stage. Di seguito è riportata l'organizzazione dei capitoli:
% \begin{description}
%     \item[{\hyperref[chap:processi-metodologie]{Capitolo 2 - Processi e metodologie}}] Descrive i processi e le metodologie adottate durante lo stage, inclusa la raccolta e la preparazione dei dati, nonché le tecniche di machine learning impiegate.
%     \item[{\hyperref[chap:descrizione-stage]{Capitolo 3 - Descrizione dello stage}}] Approfondisce gli aspetti pratici dello stage, incluse le attività svolte giornalmente e gli obiettivi specifici raggiunti.
%     \item[{\hyperref[chap:analisi-requisiti]{Capitolo 4 - Analisi dei requisiti}}] Discute i requisiti del sistema di rilevazione di malware sviluppato, evidenziando come questi siano stati identificati e come abbiano influenzato la progettazione del sistema.
%     \item[{\hyperref[chap:progettazione-codifica]{Capitolo 5 - Progettazione e codifica}}] Tratta della progettazione architetturale del sistema e della fase di codifica, descrivendo le scelte tecniche e le implementazioni specifiche.
%     \item[{\hyperref[chap:verifica-validazione]{Capitolo 6 - Verifica e validazione}}] Esamina le fasi di testing e validazione del sistema, illustrando come le soluzioni siano state testate contro i requisiti e quali metriche siano state utilizzate per valutare le performance.
%     \item[{\hyperref[chap:conclusioni]{Capitolo 7 - Conclusioni}}] Riassume i risultati ottenuti, discute le limitazioni incontrate e suggerisce possibili linee di ricerca future basate sul lavoro svolto.
% \end{description}
% Questa struttura guida il lettore attraverso le diverse fasi del progetto, evidenziando sia l'approccio teorico sia quello pratico adottato.


% Riguardo la stesura del testo, relativamente al documento sono state adottate le seguenti convenzioni tipografiche:
% \begin{itemize}
% 	\item gli acronimi, le abbreviazioni e i termini ambigui o di uso non comune menzionati vengono definiti nel glossario, situato all'inizio del presente documento;
% 	\item per la prima occorrenza dei termini riportati nel glossario viene utilizzata la seguente nomenclatura: \gls{apig};
% 	\item i termini in lingua straniera o facenti parti del gergo tecnico sono evidenziati con il carattere \textit{corsivo}.
% \end{itemize}

% % \begin{listing}[H]

% % \caption{Example of code}
% % \label{listing:a}
% % \end{listing}

% \newpage