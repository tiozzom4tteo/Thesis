\cleardoublepage
\phantomsection
\pdfbookmark{Compendio}{Compendio}
\begingroup
\let\clearpage\relax
\let\cleardoublepage\relax
\vspace*{-50pt}
\chapter*{Sommario}
\vspace*{-10pt}
Il presente lavoro di tesi si inserisce nell'ambito dell'analisi e rilevazione di malware, con un particolare focus sull'utilizzo delle Reti Generative Avversarie (GAN) come strumento per migliorare la capacità di rilevamento e comprensione di queste minacce. Durante lo stage curricolare, sono stati affrontati diversi aspetti chiave della ricerca, a partire dalla revisione della letteratura esistente sulle tecniche che combinano le GAN con l'analisi del malware. L'obiettivo primario è stato quello di progettare e sviluppare un sistema di rilevamento del malware basato su tecniche di deep learning, come le Convolutional Neural Networks (CNN), supportate da reti avanzate quali InceptionNet e XceptionNet.

Per addestrare e valutare i modelli, è stato raccolto un dataset curato di eseguibili malware da fonti affidabili come MalwareBazaar e VirusShare, etichettato tramite servizi come VirusTotal e AVClass2. I binari di malware sono stati successivamente convertiti in un formato idoneo per essere utilizzati come input per le GAN. Il modello di rete generativa avversaria sviluppato, ispirato a architetture come DCGAN e WGAN, è stato monitorato tramite metriche chiave come la Fréchet Inception Distance (FID) e la qualità visiva dei campioni generati.

Un ulteriore focus del progetto è stato l’impiego di tecniche di Explainable AI (XAI), come Grad-CAM e Lime, per migliorare la trasparenza e l’interpretabilità del modello di rilevamento. Le analisi effettuate includono una valutazione quantitativa dell'interpretabilità dei modelli e l'esecuzione di un'analisi ablation per determinare l'importanza delle caratteristiche chiave.

Il progetto fornisce contributi significativi sia in termini di innovazione nel campo della rilevazione del malware, sia nell'applicazione delle GAN come strumento per la generazione di malware sintetico e la valutazione della loro efficacia. I risultati ottenuti verranno discussi con i ricercatori coinvolti e saranno oggetto di approfondimento nella relazione finale.

\endgroup

