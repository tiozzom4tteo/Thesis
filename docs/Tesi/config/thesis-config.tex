% \omiss produces '[...]'
\newcommand{\omissis}{[\dots\negthinspace]}

% Itemize symbols
% see: https://tex.stackexchange.com/a/62497
% \renewcommand{\labelitemi}{$\bullet$}
% \renewcommand{\labelitemii}{$\cdot$}
% \renewcommand{\labelitemiii}{$\diamond$}
% \renewcommand{\labelitemiv}{$\ast$}

% Definizione delle nuove classi di titolo
\titleclass{\subsubsubsection}{straight}[\subsection]
\titleclass{\subsubsubsubsection}{straight}[\subsubsubsection]
\titleclass{\subsubsubsubsubsection}{straight}[\subsubsubsubsection] % nuovo livello

% Creazione dei nuovi contatori
\newcounter{subsubsubsection}[subsubsection]
\newcounter{subsubsubsubsection}[subsubsubsection]
\newcounter{subsubsubsubsubsection}[subsubsubsubsection] % nuovo livello

% Rinnovo dei comandi per la formattazione dei numeri delle sezioni
\renewcommand\thesubsubsubsection{\thesubsubsection.\arabic{subsubsubsection}}
\renewcommand\thesubsubsubsubsection{\thesubsubsubsection.\arabic{subsubsubsubsection}}
\renewcommand\thesubsubsubsubsubsection{\thesubsubsubsubsection.\arabic{subsubsubsubsubsection}} % nuovo livello
\renewcommand\theparagraph{\thesubsubsubsubsubsection.\arabic{paragraph}} % opzionale; utile se i paragrafi devono essere numerati

% Formattazione dei titoli delle sezioni
\titleformat{\subsubsubsection}
  {\normalfont\normalsize\bfseries}{\thesubsubsubsection}{1em}{}
\titleformat{\subsubsubsubsection}
  {\normalfont\normalsize\bfseries}{\thesubsubsubsubsection}{1em}{}
\titleformat{\subsubsubsubsubsection} % nuovo livello
  {\normalfont\normalsize\bfseries}{\thesubsubsubsubsubsection}{1em}{} 

% Spaziatura dei titoli delle sezioni
\titlespacing*{\subsubsubsection}
{0pt}{3.25ex plus 1ex minus .2ex}{1.5ex plus .2ex}
\titlespacing*{\subsubsubsubsection}
{0pt}{3.25ex plus 1ex minus .2ex}{1.5ex plus .2ex}
\titlespacing*{\subsubsubsubsubsection} % nuovo livello
{0pt}{3.25ex plus 1ex minus .2ex}{1.5ex plus .2ex}

\makeatletter
% Rinnovo dei comandi per la formattazione dei paragrafi e sottoparagrafi
\renewcommand\paragraph{\@startsection{paragraph}{6}{\z@}%
  {3.25ex \@plus1ex \@minus.2ex}%
  {-1em}%
  {\normalfont\normalsize\bfseries}}
\renewcommand\subparagraph{\@startsection{subparagraph}{7}{\parindent}%
  {3.25ex \@plus1ex \@minus .2ex}%
  {-1em}%
  {\normalfont\normalsize\bfseries}}

% Definizione dei livelli per il Table of Contents
\def\toclevel@subsubsubsection{4}
\def\toclevel@subsubsubsubsection{5}
\def\toclevel@subsubsubsubsubsection{6} % nuovo livello
\def\toclevel@paragraph{7}
\def\toclevel@subparagraph{8}

% Definizione della formattazione per il Table of Contents
\def\l@subsubsubsection{\@dottedtocline{4}{7em}{4em}}
\def\l@subsubsubsubsection{\@dottedtocline{5}{10em}{5em}}
\def\l@subsubsubsubsubsection{\@dottedtocline{6}{14em}{6em}} % nuovo livello
\def\l@paragraph{\@dottedtocline{7}{18em}{7em}}
\def\l@subparagraph{\@dottedtocline{8}{22em}{8em}}
\makeatother

% Impostazione della profondità dei numeri di sezione e del Table of Contents
\setcounter{secnumdepth}{6} % nuovo livello
\setcounter{tocdepth}{6} % nuovo livello


\let\Chaptermark\chaptermark
% \Chaptername gives current chapter name
\def\chaptermark#1{\def\Chaptername{#1}\Chaptermark{#1}}
\makeatletter
% \currentname gives the current section name
\newcommand*{\currentname}{\@currentlabelname}
\makeatother

% Uncomment the following line for a different header/footer style
% \pagestyle{fancy} \setlength{\headheight}{14.5pt}
\fancyhead[L]{\fontsize{12}{14.5} \selectfont \thechapter. \Chaptername}
\fancyhead[R]{\fontsize{12}{14.5} \selectfont \currentname}
% Page number always in footer
\cfoot{\thepage}


% Custom hyphenation rules
\hyphenation {
    e-sem-pio
    ex-am-ple
}

% Images path, not using \graphicspath because it doesn't properly work with
% latexmk custom dependencies
% \NewCommandCopy{\latexincludegraphics}{\includegraphics}
% \renewcommand{\includegraphics}[2][]{\latexincludegraphics[#1]{Tesi/images/#2}}

% Page format settings
% see: http://wwwcdf.pd.infn.it/AppuntiLinux/a2547.htm
\setlength{\parindent}{14pt}    % first row indentation
\setlength{\parskip}{0pt}       % paragraphs spacing


% Load variables
\newcommand{\myName}{Tiozzo Matteo}
\newcommand{\myID}{2042882}
\newcommand{\myTitle}{Decoding GAN-Generated Malware using Explainable AI Techniques}
\newcommand{\myDegree}{Tesi di laurea}
\newcommand{\myUni}{Università degli Studi di Padova}
% For BSc level just use "Corso di Laurea" and don't add "Triennale" to it
\newcommand{\myFaculty}{Corso di Laurea in Informatica}
\newcommand{\myDepartment}{Dipartimento di Matematica ``Tullio Levi-Civita''}
\newcommand{\profTitle}{Prof.}
\newcommand{\myProf}{Alessandro Galeazzi}
\newcommand{\tutorTitle}{Prof.}
\newcommand{\myTutor}{Vinod Puthuvath}
\newcommand{\myLocation}{Padova}
\newcommand{\myAA}{2023-2024}
\newcommand{\myTime}{Dicembre 2024}

% PDF file metadata fields
% when updating them delete the build directory, otherwise they won't change
\begin{filecontents*}{\jobname.xmpdata}
  \Title{Decoding GAN-Generated Malware using Explainable AI Techniques}
  \Author{Tiozzo Matteo}
  \Language{it-IT}
  \Subject{Il presente documento descrive il lavoro svolto durante il periodo di stage, della durata
  complessiva di trecento ore, dal laureando Tiozzo Matteo presso l'Università degli studi di Padova. Il tirocinio è stato condotto sotto la guida del Prof. Alessandro Galeazzi e con la collaborazione del Prof. Vinod Puthuvath. Il Prof. Alessandro Brighente ha
  ricoperto il ruolo di tutor accademico e referente interno al Consiglio del Corso di Studio}
  \Keywords{GAN\sep Malware Analysis}
\end{filecontents*}


% Definizione degli acronimi

% Definizione del glossario (senza duplicati)
\newglossaryentry{cybersecurity} {
    name={Cybersecurity},
    text={cybersecurity},
    sort={cybersecurity},
    description={L'insieme delle azioni volte a difendere computer, server, dispositivi mobili, sistemi elettronici, reti e dati dalle minacce informatiche}
}

\newglossaryentry{overfitting} {
    name={Overfitting},
    text={overfitting},
    sort={overfitting},
    description={ Una condizione in cui un modello di machine learning si adatta troppo ai dati di addestramento, riducendo le prestazioni sui dati di test.}
}

\newglossaryentry{gan} {
    name={Generative Adversarial Networks (GAN)},
    text={Generative Adversarial Network},
    sort={gan},
    description={Una classe di algoritmi di machine learning in cui due reti neurali, un generatore e un discriminatore, competono per migliorare le loro prestazioni.}
}

\newglossaryentry{cnn} {
    name={Convolutional Neural Network (CNN)},
    text={CNN},
    sort={cnn},
    description={Una rete neurale progettata per l'elaborazione di dati strutturati come immagini, utilizzando strati convoluzionali per estrarre caratteristiche.}
}

\newglossaryentry{tfidf} {
    name={TF-IDF (Term Frequency - Inverse Document Frequency)},
    text={TF-IDF},
    sort={tfidf},
    description={Una tecnica per quantificare l'importanza di termini in un dataset testuale, bilanciando la frequenza locale e globale.}
}

\newglossaryentry{pca} {
    name={PCA (Principal Component Analysis)},
    text={PCA},
    sort={pca},
    description={Una tecnica di riduzione della dimensionalità che identifica le componenti principali di un dataset, mantenendo la massima varianza.}
}

\newglossaryentry{minmaxscaling} {
    name={Min-Max Scaling},
    text={min-max scaling},
    sort={minmaxscaling},
    description={Una tecnica di normalizzazione dei dati che ridimensiona i valori in un intervallo specifico, spesso tra 0 e 1.}
}

\newglossaryentry{shap} {
    name={SHAP (SHapley Additive exPlanations)},
    text={SHAP},
    sort={shap},
    description={Un metodo per interpretare i modelli di machine learning, assegnando valori di contributo a ciascuna caratteristica per una predizione.}
}

\newglossaryentry{dropout} {
    name={Dropout},
    text={dropout},
    sort={dropout},
    description={Una tecnica di regolarizzazione utilizzata nelle reti neurali per prevenire l'overfitting, disattivando casualmente neuroni durante l'addestramento.}
}

\newglossaryentry{adam} {
    name={Adam Optimizer},
    text={Adam},
    sort={adam},
    description={Un algoritmo di ottimizzazione che combina i benefici di AdaGrad e RMSProp, adattando i tassi di apprendimento per ciascun parametro.}
}

\newglossaryentry{explainability}{
    name={Explainability},
    text={explainability},
    sort={explainability},
    description={La capacità di spiegare in modo chiaro e comprensibile il funzionamento e le decisioni di un modello di machine learning.}
}

\newglossaryentry{batchsize}{
    name={Batch Size},
    text={batch size},
    sort={batchsize},
    description={Il numero di campioni utilizzati per aggiornare i pesi di un modello di machine learning durante l'addestramento.}
}

\newglossaryentry{hyperparameter}{
    name={Hyperparameter},
    text={iperparametri},
    sort={hyperparameter},
    description={Un parametro che controlla il processo di addestramento di un modello di machine learning, non appreso direttamente dai dati.}
}

\newglossaryentry{lossfunction}{
    name={Loss Function},
    text={loss function},
    sort={lossfunction},
    description={Una funzione che misura l'errore tra i valori predetti e quelli reali di un modello di machine learning, utilizzata per ottimizzare i pesi.}
}

\newglossaryentry{dynamicanalysis} {
    name={Dynamic Analysis},
    text={analisi dinamica},
    sort={dynamicanalysis},
    description={Una tecnica di analisi del software che studia il comportamento di un programma durante la sua esecuzione.}
}

\newglossaryentry{staticanalysis} {
    name={Static Analysis},
    text={analisi statica},
    sort={staticanalysis},
    description={Una tecnica di analisi del software che esamina il codice sorgente o binario senza eseguire il programma.}
}

\newglossaryentry{reverseengineering} {
    name={Reverse Engineering},
    text={reverse engineering},
    sort={reverseengineering},
    description={Il processo di analisi di un sistema software o hardware per comprenderne la struttura, le funzionalità e il funzionamento.}
}

\newglossaryentry{featureextraction} {
    name={Feature Extraction},
    text={feature extraction},
    sort={featureextraction},
    description={Il processo di trasformazione dei dati grezzi in un insieme di caratteristiche utili per l'addestramento di modelli di machine learning.}
}

\newglossaryentry{confusionmatrix} {
    name={Confusion Matrix},
    text={matrice di confusione},
    sort={confusionmatrix},
    description={Una tabella utilizzata per valutare le prestazioni di un modello di classificazione, confrontando i valori predetti con quelli reali.}
}

\newglossaryentry{blackboxmodel} {
    name={Blackbox},
    text={blackbox},
    sort={blackbox},
    description={Un modello di machine learning il cui funzionamento interno non è trasparente o comprensibile, ma è valutato solo in base agli input e agli output.}
}

\newglossaryentry{datavisualization} {
    name={Data Visualization},
    text={data visualization},
    sort={datavisualization},
    description={La rappresentazione visiva di dati per facilitarne la comprensione e l'interpretazione.}
}

\newglossaryentry{hiddenlayers} {
    name={Hidden Layers},
    text={hidden layers},
    sort={hiddenlayers},
    description={Strati interni di una rete neurale che elaborano i dati tra l'input e l'output, apprendendo rappresentazioni complesse.}
}

\newglossaryentry{neurons} {
    name={Neurons},
    text={neurons},
    sort={neurons},
    description={Unità fondamentali di elaborazione in una rete neurale, responsabili dell'elaborazione e della trasmissione delle informazioni.}
}

\makeglossaries

\bibliography{appendix/bibliography}

\defbibheading{bibliography} {
    \cleardoublepage
    \phantomsection
    \addcontentsline{toc}{chapter}{\bibname}
    \chapter*{\bibname\markboth{\bibname}{\bibname}}
}

% Spacing between entries
\setlength\bibitemsep{1.5\itemsep}

\DeclareBibliographyCategory{opere}
\DeclareBibliographyCategory{web}

\addtocategory{opere}{womak:lean-thinking}
\addtocategory{web}{site:agile-manifesto}

\defbibheading{opere}{\section*{Riferimenti bibliografici}}
\defbibheading{web}{\section*{Siti Web consultati}}


\captionsetup{
    tableposition=top,
    figureposition=bottom,
    font=small,
    format=hang,
    labelfont=bf
}

\hypersetup{
    %hyperfootnotes=false,
    %pdfpagelabels,
    colorlinks=true,
    linktocpage=true,
    pdfstartpage=1,
    pdfstartview=,
    breaklinks=true,
    pdfpagemode=UseNone,
    pageanchor=true,
    pdfpagemode=UseOutlines,
    plainpages=false,
    bookmarksnumbered,
    bookmarksopen=true,
    bookmarksopenlevel=1,
    hypertexnames=true,
    pdfhighlight=/O,
    %nesting=true,
    %frenchlinks,
    urlcolor=webbrown,
    linkcolor=RoyalBlue,
    citecolor=webgreen
    %pagecolor=RoyalBlue,
}

% Delete all links and show them in black
\if \isprintable 1
    \hypersetup{draft}
\fi
\definecolor{vscodeGreen}{RGB}{76, 175, 80}
% Listings setup
% \lstset{
%     language=[LaTeX]Tex,%C++,
%     keywordstyle=\color{RoyalBlue}, %\bfseries,
%     basicstyle=\scriptsize\ttfamily,
%     %identifierstyle=\color{NavyBlue},
%     commentstyle=\color{vscodeGreen}\ttfamily,
%     stringstyle=\rmfamily,
%     numbers=none, %left,%
%     numberstyle=\scriptsize
%     stepnumber=5,
%     numbersep=8pt,
%     showstringspaces=false,
%     breaklines=true,
%     frameround=false,
%     frame=false
% }
\lstset{
    language=Python,
    basicstyle=\ttfamily\small,
    keywordstyle=\color{violet},
    commentstyle=\color{RoyalBlue},
    stringstyle=\color{vscodeGreen},
    numberstyle=\tiny\color{gray},
    stepnumber=1,
    numbersep=5pt,
    showstringspaces=false,
    breaklines=true,
    frame=single,
    captionpos=b
}

\definecolor{webgreen}{rgb}{0,.5,0}
\definecolor{webbrown}{rgb}{.6,0,0}

\newcommand{\sectionname}{sezione}
\addto\captionsitalian{\renewcommand{\figurename}{Figura}
                       \renewcommand{\tablename}{Tabella}}

\newcommand{\glsfirstoccur}{\ap{{[g]}}}

\newcommand{\intro}[1]{\emph{\textsf{#1}}}

% Risks environment
\newcounter{riskcounter}                % define a counter
\setcounter{riskcounter}{0}             % set the counter to some initial value

%%%% Parameters
% #1: Title
\newenvironment{risk}[1]{
    \refstepcounter{riskcounter}        % increment counter
    \par \noindent                      % start new paragraph
    \textbf{\arabic{riskcounter}. #1}   % display the title before the content of the environment is displayed
}{
    \par\medskip
}

\newcommand{\riskname}{Rischio}

\newcommand{\riskdescription}[1]{\textbf{\\Descrizione:} #1.}

\newcommand{\risksolution}[1]{\textbf{\\Soluzione:} #1.}

% Use case environment
\newcounter{usecasecounter}             % define a counter
\setcounter{usecasecounter}{0}          % set the counter to some initial value

%%%% Parameters
% #1: ID
% #2: Nome
\newenvironment{usecase}[2]{
    \renewcommand{\theusecasecounter}{\usecasename #1}  % this is where the display of
                                                        % the counter is overwritten/modified
    \refstepcounter{usecasecounter}             % increment counter
    \vspace{10pt}
    \par \noindent                              % start new paragraph
    {\large \textbf{\usecasename #1: #2}}       % display the title before the
                                                % content of the environment is displayed
    \medskip
}{
    \medskip
}

\newcommand{\usecasename}{UC}

\newcommand{\usecaseactors}[1]{\textbf{\\Attori Principali:} #1. \vspace{4pt}}
\newcommand{\usecasepre}[1]{\textbf{\\Precondizioni:} #1. \vspace{4pt}}
\newcommand{\usecasedesc}[1]{\textbf{\\Descrizione:} #1. \vspace{4pt}}
\newcommand{\usecasepost}[1]{\textbf{\\Postcondizioni:} #1. \vspace{4pt}}
\newcommand{\usecasealt}[1]{\textbf{\\Scenario Alternativo:} #1. \vspace{4pt}}

% Namespace description environment
\newenvironment{namespacedesc}{
    \vspace{10pt}
    \par \noindent  % start new paragraph
    \begin{description}
}{
    \end{description}
    \medskip
}

\newcommand{\classdesc}[2]{\item[\textbf{#1:}] #2}

\titleformat{\paragraph}
{\normalfont\normalsize\bfseries} 
{}                          
{0.5em}                       
{}          