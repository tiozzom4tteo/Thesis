\cleardoublepage
\phantomsection
\pdfbookmark{Sommario}{Sommario}
\begingroup
\let\clearpage\relax
\let\cleardoublepage\relax
\let\cleardoublepage\relax

\chapter*{Sommario}

Il presente documento descrive il lavoro svolto durante il periodo di stage, della durata complessiva di trecento ore, dal laureando Matteo Tiozzo presso l'Università degli studi di Padova.
Il tirocinio è stato condotto sotto la guida del Prof. Alessandro Galeazzi e con la collaborazione del Prof. Vinod Puthuvath.
Il Prof. Alessandro Brighente ha ricoperto il ruolo di tutor accademico e referente interno al Consiglio del Corso di Studio.
\\\\
Questa tesi riguarda l'analisi di malware attraverso le Generative Adversarial Network (GAN) nel contesto di malware moderni con la particolare attenzione alle famiglie di malware Windows. 
Lo studio ha esplorato i pattern che i malware possiedono in modo da poter effettuare un riconoscimento approfondito del malware. 
\\\\
Il tirocinio si è suddiviso in due parti.
La prima dedicata alla preparazione del dataset di malware, con conseguente scaricamento dello stesso, suddivisione e analisi dei pattern e caratteristiche di ciascuna famiglia di malware, nonché alla generazione delle immagini in scala di grigi per poter addestrare le GAN. La seconda parte consiste nell'analisi delle tecnologie GAN e nella loro attuale implementazione per la generazione e individuazione di malware. Inoltre la seconda parte ha anche provveduto all'addestramento delle GAN e alla valutazione dei risultati ottenuti.
%\vfill

%\selectlanguage{english}
%\pdfbookmark{Abstract}{Abstract}
%\chapter*{Abstract}

%\selectlanguage{italian}

\endgroup

\vfill
