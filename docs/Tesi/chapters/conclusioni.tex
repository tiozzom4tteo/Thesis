\chapter{Conclusioni e Sviluppi Futuri}
\label{cap:conclusioni}
\section{Consuntivo finale}
Questo studio ha esplorato l'efficacia delle Generative Adversarial Networks (GAN) nel contesto della rilevazione, classificazione e generazione di malware, con un focus specifico sulle capacità del generatore di eludere il discriminatore. L'analisi dettagliata ha permesso di confermare le ipotesi iniziali, dimostrando che le GAN possono efficacemente identificare e classificare le diverse famiglie di malware presenti nel dataset utilizzato.
Importante è stato il riscontro sulla capacità del generatore di incrementare il livello di rumore nelle immagini in scala di grigi per eludere il riconoscimento del discriminatore. Questo ha evidenziato non solo la vulnerabilità dei sistemi di rilevamento basati su apprendimento automatico, ma anche la potenzialità delle GAN nel generare varianti di malware che possono bypassare i meccanismi di sicurezza tradizionali.
Le implicazioni di queste scoperte sono significative per la sicurezza informatica. Sottolineano la necessità di continuare a sviluppare e affinare gli algoritmi di apprendimento automatico per tenere il passo con le tecniche sofisticate di evasione. Inoltre, i risultati suggeriscono che l'impiego di GAN potrebbe rivoluzionare il modo in cui gli esperti di sicurezza approcciano la creazione di sistemi di difesa, promuovendo un modello in cui i sistemi di rilevazione sono costantemente testati e migliorati contro attacchi generati artificialmente.

\section{Sviluppi Futuri}
~\\
\indent Nonostante il presente studio si sia soffermato principalmente su come un semplice modello possa riconoscere le famiglie di malware presenti su un dataset molto piccolo
[aggiungere qualche possibile sviluppo futuro che ti piacerebbe implementare]
