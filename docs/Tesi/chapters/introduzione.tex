\chapter{Introduzione}
\label{cap:introduzione}

\section{Motivazione}
%TODO aggiungere dettagli e ricontrollare
L'avvento della tecnologia ha cambiato radicalmente il modo in cui le persone interagiscono, lavorano e accedono alle informazioni. 
Tuttavia, l'aumento dell'utilizzo di queste tecnologie ha portato ad una crescita esponenziale delle minacce informatiche e dell'utilizzo improprio che malintenzionati possono farne. I malware hanno visto un aumento significativo negli ultimi anni, arrivando anche ad essere impiegati in contesti di guerra informatica. 
In questo contesto, è fondamentale sviluppare nuovi strumenti e tecniche per rilevare e contrastare queste minacce. Questo studio si propone di analizzare l'utilizzo delle Generative Adversarial Network (GAN) per il riconoscimento e l'analisi di malware sempre più evoluti, con particolare attenzione alle famiglie di malware Windows. È potenzialmente espandibile a qualsiasi tipologia di malware e a qualsiasi sistema operativo.


\section{Organizzazione del testo}
\indent Di seguito, viene presentata la struttura del documento :
\begin{description}
    \item[{\hyperref[cap:RelatedWorks]{Il secondo capitolo}}] presenta quanto trovato di simile nella letteratura attuale;

    \item[{\hyperref[cap:descrizione]{Il terzo capitolo}}] approfondisce il background e illustra l'idea del progetto;
    
    \item[{\hyperref[cap:processi-metodologie]{Il quarto capitolo}}] descrive dettagliatamente l'ambiente di sviluppo e presenta i singoli esperimenti condotti;

    \item[{\hyperref[cap:risultati]{Il quinto capitolo}}] riassume e analizza i risultati ottenuti dagli esperimenti;
    
    \item[{\hyperref[cap:conclusioni]{Il sesto capitolo}}] presenta le conclusioni del lavoro e propone possibili sviluppi futuri.
\end{description}
