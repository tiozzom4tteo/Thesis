\chapter{Introduzione}
\label{cap:introduzione}

\section{Motivazione}
L'avanzamento tecnologico ha trasformato radicalmente il modo in cui interagiamo, lavoriamo e accediamo alle informazioni. Nonostante i numerosi benefici, questa evoluzione ha anche accelerato la proliferazione di minacce informatiche, ampliando le opportunità di uso improprio da parte di attori malintenzionati. I malware, in particolare, hanno registrato un incremento significativo, venendo impiegati persino nei contesti di guerra informatica.
Di fronte a queste sfide, diventa cruciale sviluppare strumenti avanzati e metodologie efficaci per identificare e neutralizzare tali pericoli. Questo studio mira a esplorare l'applicazione delle Generative Adversarial Networks (GAN) per il riconoscimento e l'analisi di forme di malware sempre più sofisticate, concentrando l'attenzione sulle famiglie di malware che colpiscono i sistemi operativi Windows. Questa ricerca ha il potenziale di estendersi all'analisi di diverse tipologie di malware e sistemi operativi, offrendo così un contributo significativo alla sicurezza informatica.


\section{Organizzazione del testo}
\indent Di seguito, viene presentata la struttura del documento:
\begin{description}
    \item[{\hyperref[cap:RelatedWorks]{Il secondo capitolo}}] presenta quanto trovato di simile nella letteratura attuale.

    \item[{\hyperref[cap:background]{Il terzo capitolo}}] fornisce le basi teoriche e concettuali necessarie per comprendere le metodologie e le tecniche utilizzate nell'analisi del malware con le Generative Adversarial Networks (GAN).

    \item[{\hyperref[cap:descrizione]{Il quarto capitolo}}] approfondisce il background e illustra l'idea del progetto.
    
    \item[{\hyperref[cap:processi-metodologie]{Il quinto capitolo}}] descrive dettagliatamente l'ambiente di sviluppo e presenta i singoli esperimenti condotti.

    \item[{\hyperref[cap:risultati]{Il sesto capitolo}}] riassume e analizza i risultati ottenuti dagli esperimenti.
    
    \item[{\hyperref[cap:conclusioni]{Il settimo capitolo}}] presenta le conclusioni del lavoro e propone possibili sviluppi futuri.
\end{description}
