\chapter{Introduzione}
\label{cap:introduzione}

\section{Motivazione}
Negli ultimi anni, il panorama delle minacce informatiche ha subito un'evoluzione significativa, con attacchi malware sempre più sofisticati e distruttivi. Episodi come WannaCry, NotPetya, Stuxnet, Mirai e l'attacco a Travelex hanno evidenziato la vulnerabilità delle infrastrutture digitali globali, causando danni economici ingenti e interruzioni operative su larga scala.
Per contrastare efficacemente queste minacce in continua evoluzione, è fondamentale sviluppare strumenti avanzati e metodologie efficaci per identificare e neutralizzare tali pericoli. In questo contesto, le Generative Adversarial Networks (GAN) emergono come una soluzione promettente. Le GAN, composte da un generatore e un discriminatore in competizione, possono essere utilizzate per generare campioni di malware realistici. Questi campioni sintetici arricchiscono i dataset di addestramento, migliorando la capacità dei modelli di machine learning di riconoscere e classificare varianti di malware, inclusi quelli sconosciuti o zero-day.
L'adozione delle GAN nel rilevamento del malware offre diversi vantaggi:
\begin{itemize} 
    \item \textbf{Miglioramento della robustezza dei modelli}: L'esposizione a una gamma più ampia di esempi durante l'addestramento consente ai modelli di identificare con maggiore precisione comportamenti malevoli. 
    \item \textbf{Adattamento alle nuove minacce}: Le GAN possono generare varianti di malware che simulano tecniche di offuscamento avanzate, preparando i sistemi di difesa a riconoscere e bloccare nuove minacce. 
    \item \textbf{Riduzione della dipendenza da dataset etichettati}: La generazione di campioni sintetici allevia la necessità di grandi quantità di dati etichettati, spesso difficili da ottenere. 
\end{itemize}
In conclusione, l'integrazione delle GAN nelle strategie di cybersecurity rappresenta un passo avanti significativo nella protezione contro le minacce informatiche moderne. Questa tecnologia offre un approccio proattivo e adattivo, essenziale per affrontare l'evoluzione continua del panorama delle minacce.


\section{Organizzazione del testo}
\indent Di seguito, viene presentata la struttura del documento:
\begin{description}
    \item[{\hyperref[cap:RelatedWorks]{Il secondo capitolo}}] presenta quanto trovato di simile nella letteratura attuale.

    \item[{\hyperref[cap:background]{Il terzo capitolo}}] fornisce le basi teoriche e concettuali necessarie per comprendere le metodologie e le tecniche utilizzate nell'analisi del malware con le Generative Adversarial Networks (GAN).

    \item[{\hyperref[cap:descrizione]{Il quarto capitolo}}] approfondisce il background e illustra l'idea del progetto.
    
    \item[{\hyperref[cap:processi-metodologie]{Il quinto capitolo}}] descrive dettagliatamente l'ambiente di sviluppo e presenta i singoli esperimenti condotti.

    \item[{\hyperref[cap:risultati]{Il sesto capitolo}}] riassume e analizza i risultati ottenuti dagli esperimenti.
    
    \item[{\hyperref[cap:conclusioni]{Il settimo capitolo}}] presenta le conclusioni del lavoro e propone possibili sviluppi futuri.
\end{description}
