\chapter{Risultati}
\label{cap:risultati}

In questo capitolo si presenta un'analisi dettagliata dei risultati ottenuti dagli esperimenti descritti in precedenza.

\section{Risultati degli Esperimenti GAN}
\subsection{Risultati Esperimento 1}
~\\
% \indent In questa sezione si analizzano i risultati ottenuti dall'esperimento in cui si è simulato un \emph{server web QUIC} con un comportamento modificato, vedi sezione \ref{esperimento1}.
% Il \emph{server} è configurato per operare come se non ricevesse mai conferme dei pacchetti inviati, mantenendo al contempo un \emph{PTO} impostato a 0. 
% Di seguito, vengono presentati i risultati emersi accompagnati dai relativi grafici. I dati in versione tabellare sono presenti nell'Appendice.
% \\\\
% In condizioni standard, la trasmissione ha generato un traffico di circa 5.8 \emph{Mb} con un totale di circa 4100 pacchetti scambiati. 
% Questi dati rappresentano il punto di riferimento per valutare l'impatto dei cambiamenti al comportamento del \emph{server}.
% \\\\
% Come illustrato in Figura \ref{grafico12}, che mostra il confronto del consumo dati tra i vari esperimenti e lo scenario standard, le modifiche hanno prodotto effetti considerevolmente diversi.
% Negli scenari di \emph{Retransmission 1 e 2} si è osservato un incremento del traffico dati di più del $100\%$, passando dai 5.8 \emph{Mb} dello scenario standard a circa 13 \emph{Mb} nel primo caso e 15 \emph{Mb} nel secondo. 
% Parallelamente, come viene evidenziato in Figura \ref{grafico1}, il numero di pacchetti scambiati è aumentato da 4100 a circa 18000. 
% Tuttavia, questi scenari hanno presentato problemi di congestione e latenza che ne hanno compromesso l'applicabilità pratica.
% \\\\
% Le varianti 4 e 5 hanno mostrato un comportamento differente, principalmente grazie alla modifica che permette l'accettazione di un \emph{ACK} ogni due ricevuti.
% La variante 4, che non ignora la richiesta di chiusura della connessione, ha registrato un consumo di 58 \emph{Mb} e circa 65000 pacchetti, senza causare latenza o congestioni. 
% La variante 5, che invece blocca le richieste di chiusura della connessione, ha mostrato un consumo di 29 \emph{Mb} e circa 35000 pacchetti, manifestando una latenza minima nel caricamento dei contenuti.

% \begin{figure}[!h]
%     \centering
%     \begin{minipage}{0.48\textwidth}
%         \centering
%         % \includegraphics[width=\textwidth]{graphTraffico1.pdf}
%         \caption{\emph{Traffico Dati (Mb)}}
%         \subcaption*{Consumo totale del traffico dati di ogni esperimento in confronto alla connessione standard.}
%         \label{grafico12}
%     \end{minipage}
%     \hfill
%     \begin{minipage}{0.48\textwidth}
%         \centering
%         % \includegraphics[width=\textwidth]{graphNumPacchetti1.pdf}
%         \caption{\emph{Pacchetti Trasmessi}}
%         \subcaption*{Numero totale di pacchetti inviati in una connessione per ogni esperimento in confronto alla connessione normale.}
%         \label{grafico1}
%     \end{minipage}
% \end{figure}
% \noindent I risultati evidenziano come la gestione degli \emph{ACK} unita ad un \emph{PTO} costante a 0 influenzi significativamente sia il consumo di dati che il numero di pacchetti scambiati.
% La variante 4 in particolare, incrementa il traffico dati di dieci volte, passando da 5.8 Mb a 58 Mb, e il numero di pacchetti da 4100 a 65500, senza però introdurre latenza o congestione. 
% Tutto questo implica che, pur mantenendo una connessione all'apparenza normale, le modifiche comportino un potenziale aumento significativo del traffico contabilizzato per l'utente.
% \subsection{Risultati Esperimento 2}
% ~\\
% \indent In questa sezione si analizzano i risultati ottenuti dall'esperimento in cui si è simulato un \emph{server web QUIC} che inietta pacchetti aggiuntivi non richiesti in background, vedi sezione \ref{esperimento2}.
% Il \emph{server} è stato configurato per mantenere una connessione \emph{QUIC} standard con il \emph{client}, mentre contemporaneamente invia pacchetti aggiuntivi non correlati alla comunicazione principale.
% \\\\
% I risultati dell'esperimento illustrati nelle Figure \ref{grafico22} e \ref{grafico2}, mostrano gli effetti dell'iniezione di pacchetti sul consumo di dati e sul volume del traffico in termini di pacchetti scambiati.
% Come punto di riferimento si ha che la connessione \emph{QUIC} standard genera un traffico di circa 5.8 \emph{Mb} con 4100 pacchetti scambiati.
% \\\\
% L'introduzione dell'iniezione di pacchetti ha portato a un aumento prevedibile sia nel volume di traffico che nel numero di pacchetti.
% Nel primo scenario \emph{(Inject - 1)}, con 6 \emph{worker} che inviano 1000 pacchetti al secondo, si è ottenuto un aumento significativo del traffico raggiungendo circa 272 \emph{Mb} e quasi 190'000 pacchetti scambiati in 30 secondi. 
% Aumentando la frequenza di invio a 2000 pacchetti al secondo \emph{(Inject - 2)} e mantendo sempre a 6 i \emph{workers}, il traffico è quasi raddoppiato, arrivando a circa 566 \emph{Mb} con 396'000 pacchetti.
% \\\\
% Nel terzo scenario \emph{(Inject - 3)}, aumentando i \emph{worker} a 8 ma mantenendo i pacchetti al secondo a 1000, si è ottenuto un aumento del consumo dati di circa 407 \emph{Mb} e 535'000 pacchetti. 
% Infine nella quarta variante \emph{(Inject - 4)}, combinando 8 \emph{worker} e una frequenza di invio di 2000 pacchetti al secondo, si è ottenuto il picco di circa 753 \emph{Mb}, con 543'000 pacchetti.
% \begin{figure}[!h]
%     \centering
%     \begin{minipage}{0.48\textwidth}
%         \centering
%         % \includegraphics[width=\textwidth]{graphTraffico2.pdf}
%         \caption{\emph{Traffico Dati (Mb)}}
%         \subcaption*{Consumo totale del traffico dati di ogni esperimento in confronto alla connessione standard in 30 secondi.}
%         \label{grafico22}
%     \end{minipage}
%     \hfill
%     \begin{minipage}{0.48\textwidth}
%         \centering
%         % \includegraphics[width=\textwidth]{graphNumPacchetti2.pdf}
%         \caption{\emph{Pacchetti Trasmessi}}
%         \subcaption*{Numero totale di pacchetti inviati in una connessione per ogni esperimento in confronto alla connessione normale in 30 secondi.}
%         \label{grafico2}
%     \end{minipage}
% \end{figure}
% \\
% \noindent Da sottolineare che l'ambiente di test è locale e via cavo. È proprio per questa ragione che non si sono sperimentati né congestione né latenza durante gli esperimenti. 
% In uno scenario di rete reale, con connessioni \emph{wireless} o su lunghe distanze, ci si aspetterebbe la presenza di latenza o congestione, soprattutto considerando gli elevati volumi di traffico generati. 
% \begin{figure}[!h]
%     \centering
%         % \includegraphics[width=\textwidth]{graphTrafficoTempo.pdf}
%         \caption{\emph{Andamento del Consumo Dati nel Tempo}}
%         \subcaption*{Andamento del consumo dati di ogni esperimento in confronto alla connessione normale in 30 secondi.}
%         \label{grafico23}
% \end{figure}
% \\\\
% \noindent A differenza dei risultati dell'esperimento 1, in questo caso il volume di traffico risulta costante.
% Mentre nel primo scenario il volume era influenzato dalla variabile del peso della connessione principale, in questo esperimento tale fattore diventa irrilevante,
% poichè i pacchetti vengono inviati con una frequenza costante, indipendentemente dagli altri parametri.
% \\\\
% Ciò che assume particolare rilevanza è la durata in cui la connessione rimane attiva, poiché essa determina la quantità totale di pacchetti che possono essere inviati.
% Nell'esperimento che si è condotto, la connessione è stata simulata per una durata di 30 secondi. 
% La Figura \ref{grafico23} illustra l'andamento dei singoli esperimenti a paragone con quello della connessione standard nel medesimo intervallo di tempo.

% \subsection{Risultati Esperimento 3}
% ~\\
% \indent In questa sezione si analizzano i risultati ottenuti dall’esperimento in cui si è simulato
% uno scenario di attacco in cui un attaccante esterno cerca di manipolare il traffico di una connessione \emph{QUIC} tra un \emph{client} e \emph{server}, vedi sezione \ref{esperimento3}.
% L'attacco si concentra sull'oscuramento selettivo di pacchetti con lo scopo di causare ritrasmissioni e aumentare il traffico.
% \\\\
% I risultati dell'esperimento, riportati nelle Figure \ref{grafico3} e \ref{grafico32}, mostrano l'efficacia del pattern di selezione applicato per determinare quali pacchetti oscurare. 
% Nel primo scenario \emph{(Spurious Retransmission - 1)}, l'attacco è stato eseguito bloccando tutti i pacchetti nella connessione inferiori a 85 \emph{byte}, avendo come effetto un totale blocco della risorsa.
% Ciò ha impedito qualsiasi accesso al \emph{server web}, nonostante questo sia un risultato significativo per un possibile attaccante non è rilevante per lo scopo di questo studio che invece cerca di aumentare il traffico.
% \\\\
% Nel secondo scenario \emph{(Spurious Retransmission - 2)} si è oscurato un pacchetto ogni due con dimensioni inferiori a 85 \emph{byte}. Ciò ha portato a un incremento di circa il 10\% nel numero di pacchetti, senza tuttavia creare alcuna latenza o congestione.
% Aumentando la frequenza di oscuramento a due ogni tre \emph{(Spurious Retransmission - 3)} si è causato un aumento del numero di pacchetti del 30\% rispetto alla connessione normale, con un consumo totale di 6.6 \emph{Mb} e 5377 pacchetti scambiati.
% \begin{figure}[ht]
%     \centering
%     \begin{minipage}{0.48\textwidth}
%         \centering
%         % \includegraphics[width=\textwidth]{graphTraffico3.pdf}
%         \caption{\emph{Traffico Dati (Mb)}}
%         \subcaption*{Consumo totale del traffico dati di ogni esperimento in confronto alla connessione normale.}
%         \label{grafico3}
%     \end{minipage}
%     \hfill
%     \begin{minipage}{0.48\textwidth}
%         \centering 
%         % \includegraphics[width=\textwidth]{graphNumPacchetti3.pdf}
%         \caption{\emph{Pacchetti Trasmessi}}
%         \subcaption*{Numero totale di pacchetti inviati in una connessione per ogni esperimento in confronto alla connessione normale.}
%         \label{grafico32}
%     \end{minipage}
% \end{figure}
% Nel quarto e ultimo scenario \emph{(Spurious Retransmission - 4)} sono stati oscurati quattro pacchetti su cinque, sempre selezionando quelli inferiori a 85 \emph{byte}.
% Questo ha portato un aumento del numero di pacchetti quasi del 50\%, con un totale di poco meno di 8 \emph{Mb} e 6200 pacchetti scambiati.