\chapter{Related Works}
\label{cap:RelatedWorks}
\Citeauthor{article:migan}\cite{article:migan}, nel 2019, hanno sviluppato una versione migliorata di MalGAN, denominata Improved MalGAN, che mira a eludere i rilevatori di malware incorporando caratteristiche di software legittimi, ovvero cleanware, nelle funzionalità di malware. Questa ricerca, presentata al Muroran Institute of Technology, utilizza un approccio basato su Generative Adversarial Networks (GAN) per generare quantità di caratteristiche che vengono classificate come benign da un rilevatore di malware, mentre conservano la funzionalità del malware originale. L'approccio proposto migliora significativamente le tecniche esistenti impiegando un metodo di apprendimento differenziato che utilizza una singola istanza di malware, a differenza di approcci precedenti che necessitavano di multiple istanze. Questo metodo consente non solo un addestramento più realistico e praticabile per gli attaccanti, ma offre anche prestazioni migliorate in termini di evasione dai sistemi di rilevamento basati su machine learning. I risultati sperimentali indicano che l'Improved MalGAN supera le configurazioni precedenti in termini di performance di evasione, evidenziando la vulnerabilità dei metodi di rilevamento del malware esistenti e la necessità di continuare a sviluppare contromisure più robuste.\\\\
Pubblicato su Computers \& Security nel 2023, \Citeauthor{article:computersecurity}\cite{article:computersecurity} hanno esplorato l'uso delle Generative Adversarial Networks (GAN) per bypassare le tecniche di rilevazione del malware basate sulla visualizzazione. Queste tecniche, che trasformano file eseguibili in immagini per l'analisi tramite deep learning, risultano vulnerabili alle strategie avanzate di obfuscamento implementate tramite GAN. Gli autori hanno sviluppato una GAN che modifica varianti di malware per nascondere i pattern rilevabili, raggiungendo un tasso di successo del 100\% nel loro dataset testato, sottolineando la necessità di evolvere continuamente le tecnologie di rilevazione del malware per affrontare minacce in evoluzione. Questo lavoro evidenzia le vulnerabilità delle tecniche di visualizzazione attuali e stimola lo sviluppo di sistemi più robusti e avanzati.\\\\
Nel 2024, pubblicato dall'IEEE, \Citeauthor{article:DeepLearningForCyberthreats}\cite{article:DeepLearningForCyberthreats} hanno esplorato l'applicazione delle Deep Neural Networks (DNN) per la classificazione di malware nel contesto del computing perimetrale, dove i dispositivi sono spesso limitati in termini di risorse computazionali. L'articolo analizza diverse architetture DNN, come ResNet, DenseNet, InceptionNet, Xception e MobileNet, evidenziando come ciascuna di queste possa influenzare l'efficienza e l'efficacia della classificazione di malware in tempo reale. Questo studio dimostra il potenziale delle DNN nel migliorare notevolmente la precisione e l'efficacia della rilevazione di malware, classificando con alta accuratezza vari tipi di malware e sottolineando l'importanza di ottimizzare le prestazioni per il loro utilizzo in dispositivi con risorse limitate. La ricerca sottolinea l'importanza di un rilevamento precoce del malware per prevenire conseguenze negative, e propone un'innovativa distribuzione di compiti di sicurezza su dispositivi periferici per mantenere l'integrità e la disponibilità dei sistemi IoT su larga scala. Inoltre, viene discussa l'applicabilità delle DNN in dispositivi edge, considerando il tempo di computazione e la latenza nella classificazione dei binari di malware, con l'obiettivo di avanzare le procedure di cybersecurity e di rendere gli ecosistemi digitali più resilienti e sicuri contro le minacce cyber sempre più crescenti.\\\\
Nel 2024, \Citeauthor{article:ExpertSystems}\cite{article:ExpertSystems} hanno introdotto un'innovativa architettura di Generative Adversarial Network (GAN) chiamata MIGAN, per la sintesi di immagini di malware e la classificazione migliorata su un nuovo dataset. MIGAN supera le tradizionali tecniche di riconoscimento malware basate sul machine learning che richiedono notevoli competenze di dominio o analisi comportamentale dispendiosa in termini di tempo. Il framework proposto include una rete generatrice e discriminante accoppiata a un modulo di classificazione. La novità risiede nella struttura della rete GAN, nella funzione di perdita ibrida, nel nuovo dataset e nella struttura della rete di classificazione. Gli autori hanno generato circa 50.000 immagini di malware sintetiche, con le quali hanno addestrato due reti di classificazione: una rete di classificazione personalizzata e una rete Resnet50v2 preaddestrata utilizzando un approccio di transfer learning. I risultati dimostrano l'efficacia del framework, raggiungendo un'Area Under the Curve (AUC) del 99.2\%, un punteggio F1 del 99.3\% e un'accuratezza del 99.5\%. Inoltre, il sistema ha mostrato di poter mitigare efficacemente il problema dell'imbilanciamento delle classi nei dataset Malimg e in quelli personalizzati. Questo studio non solo affronta il problema dell'imbilanciamento delle classi ma apre anche nuove prospettive per lo sviluppo di sistemi di riconoscimento e classificazione di malware efficienti, a basso costo e altamente affidabili per ambienti non controllati.\\\\