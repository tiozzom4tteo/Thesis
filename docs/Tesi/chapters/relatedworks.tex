\chapter{Related Works}
\label{cap:RelatedWorks}
%\Citeauthor{article:handshake} per citare l'autore di un articolo
Questo capitolo offre un'esplorazione completa delle metodologie avanzate nel machine learning e nel deep learning, specificamente applicate alla classificazione delle immagini di malware. Esamineremo due aree principali: gli algoritmi sofisticati del deep learning e le tecniche più tradizionali ma efficaci del machine learning. Comprendendo le peculiarità e le potenzialità di entrambi gli approcci, i lettori acquisiranno preziose conoscenze sulle strategie utilizzate per identificare, analizzare e classificare il malware attraverso dati visivi, creando una solida base per le discussioni future. Inoltre, il capitolo analizza in dettaglio gli avanzamenti più recenti nella classificazione delle immagini di malware, evidenziando le lacune e le sfide attuali nel campo. I capitoli successivi approfondiranno questi temi, esaminando le soluzioni sviluppate nella tesi.

\section{Machine Learning}
L'introduzione delle tecniche di machine learning si è dimostrata fruttuosa in vari domini e l'uso nella cybersecurity è in aumento [69].

Nel 2011, Nataraj et al. [49] hanno introdotto l'analisi del malware e l'uso di tecniche di visualizzazione, accompagnandola con risultati apprezzabili nella classificazione. Questo studio ha utilizzato un algoritmo per trasformare il file binario dei campioni malevoli in immagini in scala di grigi. Le caratteristiche di texture GIST sono state estratte dal dataset di immagini per implementare un classificatore SVM. Questo approccio aiuta a categorizzare malware nuovi e sconosciuti, oltre a riconoscere linee di discendenza e versioni distinte all'interno delle famiglie di malware. L'accuratezza complessiva del modello è stata testata sul dataset MalImg presentato nello stesso studio. Su oltre 9339 campioni suddivisi in 25 famiglie, il classificatore ha ottenuto un'accuratezza del 97,18%.

Nel caso dell'analisi visiva del malware, un approccio moderno si trova in [47], dove gli autori utilizzano classificatori come random forests, SVM e alberi decisionali per ottenere risultati con un'accuratezza superiore al 90%, a seconda del metodo di estrazione delle caratteristiche. L'idea innovativa presentata in questo articolo è l'uso di un ensemble di tecniche di estrazione delle caratteristiche globali (GIST) e locali (D-SIFT). Essi riducono le dimensioni dei pattern attraverso l'assegnazione a cluster e la regolarizzazione per affrontare problemi computazionali. I risultati mostrano che il modello raggiunge un'accuratezza del 98% sul database Malimg, ma con un tempo di esecuzione leggermente più lento.

Fu et al. [23] hanno presentato una tecnica per visualizzare il malware sia globalmente che localmente per ottenere una classificazione precisa. Il loro approccio prevedeva la rappresentazione del malware come immagini a colori RGB ed estrazione delle caratteristiche globali tramite la Gray-level Co-occurrence Matrix per la texture e color moments per gli attributi di colore. Inoltre, hanno incorporato sequenze di byte speciali estratte localmente dalle sezioni di codice e dati, trasformando tutte queste caratteristiche in vettori di caratteristiche utilizzando Simhash. Integrando caratteristiche globali e locali, hanno abilitato una classificazione efficiente utilizzando algoritmi come random forest, K-nearest neighbor e support vector machine. Questo approccio ha portato all'accuratezza più alta del 97,47% e al F-measure più alto del 96,85% quando testato su un dataset composto da 7087 campioni provenienti da 15 diverse famiglie di malware.

In [81], gli autori hanno presentato due approcci per identificare il codice malware. La prima soluzione prevedeva la classificazione del malware attraverso un metodo di apprendimento integrato, dimostrando un'elevata accuratezza nella classificazione, in particolare con tecniche di ensemble multi-modello. La seconda soluzione affrontava la visualizzazione delle matrici di caratteristiche utilizzando l'algoritmo t-SNE per determinare il numero di famiglie di malware, seguito dal clustering utilizzando l'algoritmo k-means. Il documento ha inoltre introdotto un concetto nuovo chiamato visual character matrix, che ha determinato con successo il numero di famiglie di malware. Gli autori hanno suggerito applicazioni future di questi metodi nel rilevamento di codice malevolo in dispositivi IoT e reti di sensori wireless.

Roseline et al. [60] hanno proposto un sistema anti-malware che impiega un ensemble stratificato di random forests, imitando le tecniche di deep learning ma con prestazioni migliorate. Questo approccio non richiede la regolazione degli iperparametri o la retropropagazione e opera con una complessità del modello ridotta. Il sistema proposto ha raggiunto tassi di rilevamento del 98,65% e del 97,2% per Malimg e BIG 2015, superando altri metodi all'avanguardia.

Le tecniche di machine learning hanno dimostrato una capacità ragionevole nel rilevare e categorizzare il malware, con punti di forza notevoli nell'efficienza e nell'interpretabilità intrinseca dei loro modelli. Tuttavia, la loro utilità è stata oscurata dalla prevalenza crescente e dalle prestazioni superiori dei metodi di deep learning. Inoltre, la necessità di un'estrazione meticolosa delle caratteristiche, che spesso richiede assistenza esperta, rappresenta un'altra vulnerabilità degli approcci di machine learning.

\section{Deep Learning}
Oltre agli approcci tradizionali di machine learning, più recentemente sono state esplorate le reti neurali, in particolare le CNN, che hanno dimostrato risultati superiori rispetto ai modelli di machine learning convenzionali nell'analisi del malware.

In [26], gli autori migliorano la tecnica di visualizzazione utilizzando una CNN progettata appositamente. L'estrazione delle caratteristiche è stata eseguita tramite la rete neurale senza la necessità di altri algoritmi. Il modello complessivo ha migliorato l'accuratezza dell'1,30% rispetto al primo modello di Nataraj et al., dimostrando l'utilità delle CNN.

In [34], gli autori esplorano l'efficacia delle Convolutional Neural Networks (CNN) in questo contesto. L'approccio proposto converte i binari del malware in immagini in scala di grigi e addestra una CNN per la classificazione. Gli esperimenti condotti su dataset standard di malware (Malimg e Microsoft malware) mostrano che il metodo supera le tecniche esistenti, raggiungendo tassi di accuratezza del 98,52% e 99,97% sui rispettivi dataset. In questo caso, come nel documento precedentemente discusso di Gibert [26], la CNN proposta è stata sviluppata da zero dagli autori.

Ni et al. [51] propongono un algoritmo di classificazione del malware, Malware Classification using SimHash and CNN (MCSC), per affrontare il compito di classificazione del malware utilizzando la visualizzazione. MCSC converte i codici malware disassemblati in immagini in scala di grigi utilizzando SimHash e poi utilizza le CNN per identificare le loro famiglie. L'algoritmo incorpora multi-hash, selezione del blocco principale e interpolazione bilineare per migliorare le prestazioni. I risultati sperimentali dimostrano l'efficacia di MCSC nella classificazione delle famiglie di malware, anche con campioni distribuiti in modo non uniforme. L'algoritmo raggiunge un'accuratezza di classificazione fino al 99,26% e un'accuratezza media del 98,86% su un dataset di 10.805 campioni, superando altri algoritmi confrontati.

In [66], gli autori discutono la vulnerabilità dei dispositivi Internet of Things (IoT) agli attacchi informatici, in particolare agli attacchi Distributed Denial of Service (DDoS), a causa della mancanza di misure di sicurezza di base. Propongono un metodo leggero per rilevare malware DDoS negli ambienti IoT. L'approccio, simile alle tecniche precedentemente discusse, converte i binari del malware in immagini in scala di grigi e utilizza una rete neurale convoluzionale per la classificazione. I risultati sperimentali dimostrano che il sistema raggiunge un'accuratezza del 94,0% nel distinguere tra goodware e malware DDoS e un'accuratezza dell'81,8% nella classificazione tra goodware e due famiglie di malware prominenti.

Il modello presentato in [73] combina mining di similarità e architetture di deep learning, utilizzando varie misure di distanza per calcolare le similarità tra varianti di malware. Queste misure generano matrici di similarità e le distanze tra queste immagini aiutano a identificare le famiglie di malware. L'ottimizzazione minimale sequenziale (SMO), il kernel polinomiale normalizzato e le architetture di deep learning in una CNN bidirezionale raggiungono un'accuratezza di quasi il 99%. Il metodo proposto è computazionalmente efficiente rispetto alle tecniche di machine learning tradizionali e può essere continuamente addestrato in tempo reale per gestire nuove minacce di malware.

Nel 2020, Vasan et al. [72] hanno sperimentato una versione leggermente semplificata della CNN VGG16. Gli autori hanno utilizzato immagini RGB invece di quelle in scala di grigi applicando una semplice mappa di colori sugli originali. Hanno quindi impiegato il transfer learning e il fine-tuning per ottenere un modello competitivo. Con il transfer learning, i ricercatori riutilizzano i pesi della CNN addestrati in un altro compito simile, in questo caso, la sfida ImageNet [61], e li usano per il modello che stanno sviluppando. Il fine-tuning è un ulteriore miglioramento rispetto al transfer learning perché consente agli autori di riaddestrare la rete neurale con i nuovi pesi. Di solito, per risparmiare sui costi computazionali, vengono addestrati solo alcuni strati. Nel caso di questo studio, sono stati addestrati solo gli strati densi e i primi strati convoluzionali. L'accuratezza finale ottenuta per il dataset MalImg è del 98,82%. Nello stesso anno, gli autori in [71] hanno utilizzato una tecnica di ensemble per sfruttare la potenza di diversi estrattori di caratteristiche sotto forma di diverse CNN e altri classificatori. I modelli fine-tuned impiegati nello studio sono stati VGG16 e ResNet50. Il risultato finale di accuratezza sul dataset MalImg è stato del 99,50%, segnando un grande miglioramento rispetto ai modelli precedenti e, soprattutto, senza la necessità di procedure complicate di generazione delle immagini. In generale, i ricercatori si sono diretti in due direzioni per migliorare i risultati del loro modello. Il primo gruppo si è orientato verso il miglioramento della classificazione utilizzando classificatori con prestazioni migliori, principalmente le CNN. L'altro gruppo si è concentrato su metodologie migliori per estrarre caratteristiche e creare immagini con più informazioni. Nonostante la natura intensiva in risorse e le necessità di dati delle reti neurali profonde, esse mostrano grandi promesse per i futuri sviluppi nel campo.

In [48], Naeem et al. affrontano il problema del rilevamento del malware nell'Industrial Internet of Things (IIoT). Le tecniche tradizionali di rilevamento del malware non sono adatte per i dispositivi IIoT a causa della loro complessità computazionale e delle risorse limitate. Gli autori propongono un'architettura che integra la visualizzazione del malware in un modello di Deep Convolutional Neural Network (DCNN). La tecnica prevede la conversione di file APK e PE in immagini a colori, che il modello DCNN elabora per estrarre caratteristiche dinamiche delle immagini del malware. Il metodo proposto raggiunge un'alta accuratezza di rilevamento, con il 97,81% di accuratezza sul dataset IIoT e il 98,47% di accuratezza sul dataset Windows.

In [80], gli autori propongono un metodo di visualizzazione utilizzando Colored Label boxes (CoLab) per segnare sezioni dei file Portable Executable (PE), enfatizzando le informazioni sulla distribuzione delle sezioni nelle immagini convertite del malware. Viene sviluppato un metodo di classificazione chiamato Malware classification using CoLab image, VGG16 e Support Vector Machine (MalCVS). I risultati sperimentali utilizzando dataset di malware da Virusshare e dalla Microsoft Malware Classification Challenge dimostrano l'efficacia di MalCVS. L'approccio raggiunge accuratezze medie del 96,59% e del 98,94% sui due dataset, rispettivamente, indicando un'alta accuratezza nella classificazione del malware in famiglie.

In [68], gli autori propongono un metodo che utilizza Convolutional Neural Networks (CNN), trasformando file di byte in immagini in scala di grigi e RGB per la classificazione. Introducono una nuova tecnica chiamata B2IMG per la trasformazione dei file e un metodo di data augmentation basato su CycleGAN per gestire le dimensioni dei dati sbilanciate tra le famiglie di malware. I test su dataset BIG2015 e DumpWare10 hanno mostrato miglioramenti significativi nell'accuratezza della classificazione, raggiungendo il 99,86% per BIG2015 e il 99,60% per DumpWare10, dimostrando l'efficacia del metodo proposto.

In [64], gli autori introducono un approccio innovativo al rilevamento del malware combinando tecniche di analisi statica e dinamica. I metodi tradizionali presentano limitazioni, con l'analisi statica che è veloce ma incapace di rilevare varianti di malware offuscate e l'analisi dinamica che è efficace ma lenta e intensiva in risorse. Il metodo proposto visualizza i file PE come immagini colorate, estrae caratteristiche profonde utilizzando un modello di deep learning fine-tuned e rileva il malware utilizzando macchine a vettori di supporto (SVM). L'integrazione del deep learning e del machine learning elimina la necessità di un'estrazione estensiva delle caratteristiche e della conoscenza del dominio. La validazione sperimentale che coinvolge 12 modelli di machine learning e 15 modelli di deep learning, condotta su vari dataset di riferimento, mostra prestazioni all'avanguardia. Il framework proposto ha raggiunto un'accuratezza del 99,06% sul dataset Malimg e ha dimostrato significatività statistica attraverso metodi di test rigorosi.

In [19], Deng et al. propongono un metodo di classificazione del malware, Malware Classification based on Three-channel Visualization and Deep learning (MCTVD), che utilizza immagini di malware piccole e di dimensioni uniformi e una rete neurale convoluzionale superficiale. I risultati sperimentali dimostrano l'efficacia di MCTVD, raggiungendo un'accuratezza del 99,44% sul dataset pubblico di malware di Microsoft sotto una validazione incrociata a 10 fold.

L'attuale panorama del rilevamento basato su immagini di malware è indiscutibilmente dominato dai metodi di deep learning, in particolare dalle Convolutional Neural Networks (CNN). I ricercatori hanno ampiamente dimostrato che queste architetture complesse, prese in prestito da campi diversificati, o modelli ad-hoc più semplici, producono costantemente livelli di prestazioni all'avanguardia. Il raffinamento continuo delle CNN è in corso, con le tecniche di ensemble che emergono come approccio preferito per ottenere risultati ottimali. Contestualmente, l'esplorazione delle Recurrent Neural Networks (RNN) per la classificazione visiva del malware è in corso, con sforzi nascenti ma promettenti nei trasformatori visivi. I risultati notevoli ottenuti dalle reti neurali le hanno stabilite come l'opzione preferita sia per i ricercatori che per le aziende, sostituendo i modelli di machine learning convenzionali. Questa transizione ha indubbiamente migliorato l'efficacia della categorizzazione del malware. Tuttavia, ha anche nascosto la trasparenza intrinseca trovata nei modelli di machine learning, portando all'emergere di un approccio "black-box" nel campo.

Nonostante i successi notevoli, rimane una sfida significativa riguardo alla spiegabilità. Pochi ricercatori hanno approfondito le complessità del motivo per cui vengono fatte scelte specifiche, come la selezione delle reti neurali con le migliori prestazioni per l'ensemble. Questa mancanza di chiarezza spesso porta a situazioni in cui combinare le reti neurali con le migliori prestazioni non genera costantemente i migliori risultati finali. Il campo, quindi, si confronta con il compromesso tra prestazioni superiori e la comprensione dei processi decisionali alla base di questi modelli avanzati.

La Tabella 3.1 contiene le informazioni sulla classificazione del modello e le idee esplorate nella fase di classificazione. Vengono considerate solo le principali contribuzioni di ciascun articolo. Le colonne sono: Paper - riferimento dell'articolo; D/C - se l'articolo parla di rilevamento del malware [D] o di classificazione in famiglie [C]; DT/RF - se l'articolo utilizza una tecnica di decision tree o random forest per la classificazione; KNN - se l'articolo utilizza un classificatore KNN; SVM - se l'articolo utilizza un classificatore SVM; CNN - se l'articolo utilizza una CNN come classificatore, verrà specificata quale architettura CNN viene utilizzata se è un'architettura nota; RNN - se l'articolo utilizza una RNN come classificatore, verrà specificata quale architettura RNN viene utilizzata se disponibile; Other ML - tutte le tecniche di machine learning diverse dalle categorie KNN, DT/RF e SVM; Other DL - tutte le tecniche di deep learning che non rientrano nelle categorie CNN e RNN (es. vision transformers); Split - la suddivisione training/test utilizzata per addestrare il modello, la notazione è train/validation/test, [f] indica l'uso della validazione incrociata k-fold invece del leave one out; E - se i problemi di spiegabilità e interpretabilità sono trattati.

%TODO aggiungere riferimenti e tabella pagina 24 del tipo inglese
