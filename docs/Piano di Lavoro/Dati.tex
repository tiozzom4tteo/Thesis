%----------------------------------------------------------------------------------------
%   USEFUL COMMANDS
%----------------------------------------------------------------------------------------

\newcommand{\dipartimento}{Dipartimento di Matematica ``Tullio Levi-Civita''}

%----------------------------------------------------------------------------------------
% 	USER DATA
%----------------------------------------------------------------------------------------

% Data di approvazione del piano da parte del tutor interno; nel formato GG Mese AAAA
% compilare inserendo al posto di GG 2 cifre per il giorno, e al posto di 
% AAAA 4 cifre per l'anno
\newcommand{\dataApprovazione}{Data}

% Dati dello Studente
\newcommand{\nomeStudente}{Matteo}
\newcommand{\cognomeStudente}{Tiozzo}
\newcommand{\matricolaStudente}{2042882}
\newcommand{\emailStudente}{matteo.tiozzo.1@studenti.unipd.it}
\newcommand{\telStudente}{+ 39 3345263731}

% Dati del Tutor Aziendale
\newcommand{\nomeTutorAziendale}{Alessandro}
\newcommand{\cognomeTutorAziendale}{Galeazzi}
\newcommand{\emailTutorAziendale}{alessandro.galeazzi@unipd.it}
\newcommand{\telTutorAziendale}{}
\newcommand{\ruoloTutorAziendale}{}

% Dati dell'Azienda
\newcommand{\ragioneSocAzienda}{Università di Padova, Dipartimento di Matematica}
\newcommand{\indirizzoAzienda}{Via Trieste, 63, 35131 Padova} 
\newcommand{\sitoAzienda}{https://www.math.unipd.it/}
%\newcommand{\emailAzienda}{mail@mail.it}
%\newcommand{\partitaIVAAzienda}{P.IVA 12345678999}

% Dati del Tutor Interno (Docente)
\newcommand{\titoloTutorInterno}{Prof.}
\newcommand{\nomeTutorInterno}{Alessandro}
\newcommand{\cognomeTutorInterno}{Brighente}

\newcommand{\prospettoSettimanale}{
     % Personalizzare indicando in lista, i vari task settimana per settimana
     % sostituire a XX il totale ore della settimana
    \begin{itemize}
        \item \textbf{Prima Settimana (40 ore)}
        \begin{itemize}
        	   \item Revisione della letteratura e delle tecniche esistenti per le Reti Generative Avversarie
            \item Identificazione e scaricamento di dataset di malware
            \item Etichettatura del dati per lo sviluppo del sistema di rilevamento
        \end{itemize}
        \item \textbf{Seconda Settimana (40 ore)} 
        \begin{itemize}
            \item Pre-elaborazione e pulizia dei dati
            \item Identificazione ed estrazione delle caratteristiche rilevanti
        \end{itemize}
        \item \textbf{Terza Settimana (40 ore)} 
        \begin{itemize}
            \item Studio dei modelli di deep learning per identificazione di malware 
            \item Classificazione delle tipologie dei codici sorgenti
        \end{itemize}
        \item \textbf{Quarta Settimana (40 ore)} 
        \begin{itemize}
            \item Sviluppo di esempi di reti avversarie utilizzando GAN
            \item Creazione di un dataset per la valutazione del modello
        \end{itemize}
        \item \textbf{Quinta Settimana (40 ore)} 
        \begin{itemize}
            \item  Analisi delle caratteristiche del classificatore tramite Grad-CAM/Lime/Occlusion sensitivity
            \item  Analisi delle differenze tra malware originali e generati sinteticamente
        \end{itemize}
        \item \textbf{Sesta Settimana (40 ore)} 
        \begin{itemize}
            \item Analisi della similarità delle caratteristiche dei malware nella stessa famiglia
            \item Analisi della similarità delle caratteristiche dei malware tra famiglie diverse
        \end{itemize}
        \item \textbf{Settima Settimana (40 ore)} 
        \begin{itemize}
            \item Descrizione e visualizzazione dei risulati
            \item Confronto con i ricercatori coinvolti per discutere i risultati ottenuti 
        \end{itemize}
        \item \textbf{Ottava Settimana (20 ore)} 
        \begin{itemize}
            \item Applicazione dei feedback ricevuti
            \item Redazione documentazione e relazione finale;
        \end{itemize}
    \end{itemize}
}

% Indicare il totale complessivo (deve essere compreso tra le 300 e le 320 ore)
\newcommand{\totaleOre}{300}

\newcommand{\obiettiviObbligatori}{
	 \item \underline{\textit{O01}}: Creazione di un dataset di malware classificato per categoria
	 \item \underline{\textit{O02}}: Implementazione di un sistema per identificare i malware basato su deep learning
	 \item \underline{\textit{O03}}: Creazione di un modello avversario (GAN) per generare malware
	 \item \underline{\textit{O04}}: Valutazione delle performance del modello secondo tecniche di Explenable AI
	 \item \underline{\textit{O05}}: Valutazione delle performance del modello tramite ablation analysis
	 
}

\newcommand{\obiettiviDesiderabili}{
	 \item \underline{\textit{D01}}: Analisi esaustiva dei modelli tramite Explanable AI
	 \item \underline{\textit{D02}}: Analisi approfondita dei risultati degli esperimenti;
}

\newcommand{\obiettiviFacoltativi}{
	 \item \underline{\textit{F01}}: Implementazione di diversi modelli per identificazione di malware.
	 \item \underline{\textit{F02}}: Implementazione di diverse architetture per la generazione di malware.
}