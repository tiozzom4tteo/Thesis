\documentclass[italian,12pt]{article} %tipo di documento

%--------------variabili------------------%
\def\Title{Università degli Studi di Padova}
\def\Author{Tiozzo Matteo\\(2042882)}
\def\Scuola{Scuola di Scienze}
\def\Corso{Corso di Laurea in Informatica}
\def\Piano{Piano di Lavoro}
\def\Contatti{\textbf{Contatti}\\ \textbf{Studente:} \href{mailto:matteo.tiozzo.1@studenti.unipd.it}{matteo.tiozzo.1@studenti.unipd.it}\\ \textbf{Tutor proponente:} \href{mailto:alessandro.galeazzi@unipd.it}{alessandro.galeazzi@unipd.it} \\ \textbf{Tutor interno:} \href{mailto:alessandro.galeazzi@unipd.it}{alessandro.galeazzi@unipd.it}}
\def\Data{24/08/2024}
%-----------------------------------------%

\input{packages}
\input{subX6_section}

\linespread{1.2}
\captionsetup[table]{name=Tabella}
\geometry{headsep=1.5cm}

\renewcommand{\contentsname}{Indice}%imposto il nome dell'indice
\renewcommand\familydefault{\sfdefault}

\renewcommand{\listtablename}{Indice delle tabelle}%imposto il nome della lista tabelle
\renewcommand\familydefault{\sfdefault}

\renewcommand{\listfigurename}{Indice delle immagini}%imposto il nome della lista immagini
\renewcommand\familydefault{\sfdefault}

\let\oldthepage\thepage
\renewcommand{\thepage}{\sffamily \oldthepage}

\addto\captionsitalian{\renewcommand{\figurename}{Figura}}

\begin{document} 

\newgeometry{left=2cm,right=2cm,bottom=2.1cm,top=2.1cm}
\begin{titlepage}

    {
        \centering
        {\bfseries\Large \Title\par}
        \bigbreak
        

        \begin{tikzpicture}[remember picture,overlay]
            \clip (0,-4cm) circle (8cm) node (current page.center) {\includegraphics[width=8cm]{logo.png}};
            \clip (0,-10cm) circle (8cm) node (current page.center) {
            \includegraphics[width=8cm]{dipartimento.png}};
        \end{tikzpicture}

        \vspace{12cm}

        {\bfseries\Large \Scuola\par}
        \bigbreak
        {\bfseries\Large \Corso\par}
        \bigbreak
        \begin{center}
            \rule{\linewidth}{0.3mm}
            {\bfseries\Large \Piano\par}
            \rule{\linewidth}{0.3mm}
        \end{center}
        \vspace{1cm}
        {\Contatti\par}
        \vspace{1cm}
        {\Data\par}
    }
\end{titlepage}

\restoregeometry


\newpage

%---------------header------------------%
\pagestyle{fancy}
\fancyhead{} % pulizia degli header
\lhead{%
\begin{tikzpicture}
   \clip (0,0) circle (0.5cm);
   \node at (0,0) {\includegraphics[width=1cm]{logo.png}};
\end{tikzpicture}%
}
\chead{\vspace{\fill}\Title\vspace{\fill}}
\rhead{\vspace{\fill}\Author\vspace{\fill}}
%quad è una spaziatura
%---------------------------------------%




\newpage

\tableofcontents

\newpage

% \section{Informazioni sull'azienda}
\begin{itemize}
  \item Ragione Sociale:
  \item Sede Legale:
  \item Sede Operativa:
  \item Partita IVA: 
  \item Sito web:
  \item Email:
  \item Telefono:
\end{itemize}
%Mettere descrizione azienda


\section{Scopo dello stage}
Lo scopo di questo progetto di stage è sviluppare un modello avanzato di intelligenza artificiale specializzato nella creazione e nel riconoscimento di malware utilizzando le reti generative avversarie (Generative Adversarial Networks, GAN). Questo progetto mira a sfruttare la potenza delle GAN, che consistono in una rete generativa e una rete discriminante in competizione tra loro, per generare nuovi campioni di malware e, contemporaneamente, migliorare la capacità di rilevamento di tali minacce.
\begin{flushleft}
Lo studente contribuirà al miglioramento della sicurezza informatica attraverso l'identificazione precoce e accurata di minacce emergenti che possano compromettere la sicurezza dei dispositivi e delle reti. Il progetto non solo intende affinare le tecniche di rilevamento dei malware esistenti, ma anche esplorare nuovi approcci innovativi per anticipare possibili varianti di malware ancora sconosciute. In ultima analisi, il progetto aspira a rafforzare le difese informatiche mediante lo sviluppo di strumenti intelligenti in grado di adattarsi rapidamente all'evoluzione delle minacce digitali, rendendo l'ambiente informatico più sicuro e resiliente.
\end{flushleft}
\section{Periodo}
\begin{itemize}
    \item Data prevista di inizio: 08/09/2024
    \item Data prevista di fine: 08/11/2024
\end{itemize}
\section{Interazione tra tirocinante e tutor aziendale}
Sono pianificati incontri periodici con il tutor aziendale per discutere lo stato di avanzamento del progetto, aggiornare gli obiettivi, se necessario, e risolvere eventuali dubbi. Il tutor sarà inoltre disponibile su richiesta per rispondere a domande, fornire supporto operativo durante l'intero svolgimento del progetto e garantire che il piano di lavoro rimanga aggiornato. Le comunicazioni avverranno tramite i canali principali concordati, con la possibilità di colloqui in presenza qualora si rendesse necessario un confronto diretto.
\section{Prodotti attesi}
Lo studente sarà responsabile di documentare quotidianamente il lavoro svolto e di confrontare i progressi ottenuti con quelli previsti. Al termine del periodo di stage, dovrà redigere una relazione scritta dettagliata che illustri il percorso seguito e i risultati conseguiti. In particolare, la relazione dovrà includere:
\begin{itemize} 
  \item Le conclusioni emerse dallo studio del problema e dalla ricerca bibliografica
  \item L'analisi del corpus utilizzato durante tutto lo stage
  \item La documentazione completa dei modelli sviluppati, in modo da garantire la replicabilità dei risultati ottenuti
\end{itemize}


\section{Obiettivi}
\subsection*{Notazione}
Si farà riferimento ai requisiti secondo le seguenti notazioni:
\begin{enumerate}
  \item \textbf{\textit{O}}: obbligatorio, vincolo irrinunciabile
  \item \textbf{\textit{D}}: desiderabile, vincolo non strettaente necessario, ma che da valore aggiunto al prodotto
\end{enumerate}
Le sigle precendentemente descritte saranno seguite da un numero progressivo per identificare univocamente il requisito.

\subsection*{Obiettivi fissati}
Si prevede lo svolgimento degli obiettivi riportati sotto.
\begin{itemize}
  \item Obbligatori:
  \begin{itemize}
    \item \textbf{\textit{O1}}: studio del problema e ricerca in letteratura
    \item \textbf{\textit{O2}}: implementazione e sperimentazione di modelli di GAN
    \item \textbf{\textit{O3}}: documentazione dei modelli sviluppati, con relativo repository git
  \end{itemize}
  \item Desiderabili:
  \begin{itemize}
    \item \textbf{\textit{D1}}: 
  \end{itemize}
\end{itemize}

\section{Pianificazione del lavoro}
\subsection*{Pianificazione settimanale}
\begin{itemize}
  \item \textbf{Settimana 1 (40 ore)}
  \begin{itemize}
    \item Preparazione dell'ambiente di sviluppo
    \item Studio teorico sul problema e sull'utilizzo delle Generative Adversarial Network (GAN) in merito
  \end{itemize}

  \item \textbf{Settimana 2 (40 ore)}
  \begin{itemize}
    \item Studio teorico sul problema e sull'utilizzo delle Generative Adversarial Network (GAN) in merito
    \item Ricerca teorica delle ultime vulnerabilità riscontrate su sistema operativo iOS
  \end{itemize}

  \item \textbf{Settimana 3 (40 ore)}
  \begin{itemize}
    \item Ricerca di dataset per l'addestramento dei modelli
    \item Ricerca di modelli GAN
  \end{itemize}

  \item \textbf{Settimana 4 (40 ore)}
  \begin{itemize}
    \item Implementazione di modelli di GAN
  \end{itemize}

  \item \textbf{Settimana 5 (40 ore)}
  \begin{itemize}
    \item Sperimentazione e verifica dei modelli implementati
  \end{itemize}

  \item \textbf{Settimana 6 (40 ore)}
  \begin{itemize}
    \item Soluzione degli errori e delle problematiche riscontrate
  \end{itemize}

  \item \textbf{Settimana 7 (40 ore)}
  \begin{itemize}
    \item Documentazione dei modelli sviluppati
  \end{itemize}

  \item \textbf{Settimana 8 (20 ore)}
  \begin{itemize}
    \item Scrittura della tesi
  \end{itemize}

\end{itemize}



\section{Ripartizione delle ore}
\begin{tabular}{|c|p{14cm}|}
\hline
\textbf{Durata in ore} & \textbf{Descrizione attività} \\ \hline
75 & Formazione sul problema e sugli algoritmi di machine learning \\ \hline
150 & Implementazione degli algoritmi, ottimizzazione degli stessi e correzione di possibili errori \\ \hline
75 & Stesura documentazione e scrittura tesi \\ \hline
\textbf{Totale ore} & \textbf{300} \\ \hline
\end{tabular}

\section*{Approvazione}
Il presente piano di lavoro è stato approvato dai seguenti soggetti:

\vspace{3cm}

\begin{flushleft}
    \rule{0.6\linewidth}{0.3mm} \\

    Galeazzi Alessandro \hspace{2cm} Tutor proponente 

    \vspace{3cm}

    \rule{0.6\linewidth}{0.3mm} \\ 

    Tiozzo Matteo \hspace{3.3cm} Stagista 

    \vspace{3cm}

    \rule{0.6\linewidth}{0.3mm} \\ 

    Possibile tutor interno (Vinod)
\end{flushleft}



% \section{Valore formativo}
Con questo progetto di stage il tirocinante avrà la possibilità di sfruttare le conoscenze acquisite durante il corso di laurea in Informatica. Più precisamente, avrà l'opportunità di approfondire le proprie conoscenze nell'ambito della sicurezza informatica e dell'intelligenza artificiale, ambito in cui vuole continuare la sua carriera. Inoltre, potrà lavorare in un contesto aziendale, collaborando con professionisti del settore. Durante questo progetto di stage il tirocinante potrà acquisire competenze nell'utilizzo delle seguenti tecnologie:
\begin{itemize}
  \item Python;
  \item mettere altre tecnologie;
\end{itemize}


% \section{Svolgimento dello stage}
L'attività di stage si svolgerà presso la sede operativa dell'azienda, situata in (METTERE INDIRIZZO). Il tirocinante lavorerà a stretto contatto con il tutor aziendale e con il team di sviluppo, seguendo le indicazioni e le linee guida fornite per lo svolgimento del progetto. Qualora fosse necessario, è prevista la possibilità di lavorare in modalità smart working. L'azienda si impegna a mettere a disposizione del tirocinante tutti gli strumenti e le risorse necessarie per lo svolgimento del progetto, sia hardware che software.


\end{document}
