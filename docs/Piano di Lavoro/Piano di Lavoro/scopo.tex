\section{Scopo dello stage}
Lo scopo di questo progetto di stage è sviluppare un modello avanzato di intelligenza artificiale specializzato nella creazione e nel riconoscimento di malware utilizzando le reti generative avversarie (Generative Adversarial Networks, GAN). Questo progetto mira a sfruttare la potenza delle GAN, che consistono in una rete generativa e una rete discriminante in competizione tra loro, per generare nuovi campioni di malware e, contemporaneamente, migliorare la capacità di rilevamento di tali minacce.
\begin{flushleft}
Lo studente contribuirà al miglioramento della sicurezza informatica attraverso l'identificazione precoce e accurata di minacce emergenti che possano compromettere la sicurezza dei dispositivi e delle reti. Il progetto non solo intende affinare le tecniche di rilevamento dei malware esistenti, ma anche esplorare nuovi approcci innovativi per anticipare possibili varianti di malware ancora sconosciute. In ultima analisi, il progetto aspira a rafforzare le difese informatiche mediante lo sviluppo di strumenti intelligenti in grado di adattarsi rapidamente all'evoluzione delle minacce digitali, rendendo l'ambiente informatico più sicuro e resiliente.
\end{flushleft}