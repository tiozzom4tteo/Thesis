%----------------------------------------------------------------------------------------
%	DESCRIPTION OF THE PRODUCTS THAT ARE BEING EXPECTED FROM THE STAGE
%----------------------------------------------------------------------------------------
\section*{Prodotti attesi}
% Personalizzare definendo i prodotti attesi (facoltativo)

Lo studente dovrà produrre una relazione scritta che illustri i seguenti punti:

\begin{itemize}
    \item Revisione della letteratura esistente \\
    Effettuare una revisione della letteratura sulle tecniche che utilizzano le Reti Generative Avversarie (GAN) nel campo dell'analisi e rilevazione di malware. 
    
    \item Raccolta del dataset di malware \\
    Raccogliere un dataset curato di eseguibili malware da fonti affidabili come Malwarebazaar, Virusshare, ecc., garantendo qualità e diversità dei dati. Utilizzare servizi come VirusTotal o AVClass2 per costruire un dataset di malware etichettato per tipologia.

    \item Conversione dei binari di malware \\
    Convertire i binari di malware in un formato adatto come input per le GAN

    \item Sviluppo del sistema di rilevazione malware \\
    Progettare e implementare un sistema di rilevazione del malware utilizzando algoritmi di deep learning come Convolutional Neural Networks (CNN), InceptionNet, XceptionNet e altri.

    \item Addestramento dei modelli GAN \\
    Sviluppare un'architettura GAN (es. DCGAN, WGAN) adatta alla generazione di malware. Monitorare metriche chiave come perdita, FID (Fréchet Inception Distance) e la qualità visiva dei campioni di malware generati.

    \item Applicazione delle tecniche di Explainability \\
     Analizzare le prestazioni tramite tecniche di Explenable AI come Grad-CAM e Lime.

    \item Valutazione dell'interpretabilità (Analisi quantitativa) e Ablation analysis\\
    Misurare la coerenza delle caratteristiche evidenziate su diversi campioni e tipologie di malware. 
    Eseguire ablation analysis rimuovendo o modificando le caratteristiche chiave.

\end{itemize}
