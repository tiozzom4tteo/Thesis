\section*{Contenuti formativi previsti}
% Personalizzare indicando le tecnologie e gli ambiti di interesse dello stage
Il progetto prevede che lo studente metta in pratica e approfondisca le sue conoscenze nell'ambito dell'Intelligenza Artificiale Interpretabile (Explainable AI) e delle Reti Generative Avversarie (GAN). Inizialmente, lo studente dovrà acquisire competenze nella comprensione e manipolazione di dataset di malware, con un focus particolare sull'analisi delle caratteristiche generate dalle GAN. Successivamente, è richiesto che familiarizzi con metodi per migliorare la trasparenza e l'interpretabilità dei modelli generativi, utilizzando tecniche come Grad-CAM e Lime. Durante l'attività di stage, lo studente potrà quindi approfondire le tecniche avanzate e gli strumenti utilizzati per rendere i modelli di intelligenza artificiale più trasparenti e interpretabili nel contesto della sicurezza informatica.
\newpage